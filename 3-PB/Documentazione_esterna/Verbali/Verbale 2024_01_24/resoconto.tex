\section{Resoconto} \label{sec:resoconto}
\subsection{Discussioni} \label{subsec:resdiscussione}
\begin{enumerate}
    \item L'incontro si è aperto con l'aggiornamento sullo stato di avanzamento del progetto, focalizzandosi sull'aspetto didattico relativo alla consegna dell'RTB. \\È stato sottolineato il forte impegno del gruppo nel mantenere un ritmo costante nelle attività, anche durante il periodo di sessione d'esame, al fine di garantire che la consegna prevista per la prima settimana di aprile proceda senza rallentamenti.
    
    \item La discussione è proseguita con una rilettura cooperativa dei requisiti, conclusasi con un giudizio positivo da parte dei proponenti.
    
    \item Durante la lettura dell'Aanalidi\_dei\_requisiti\_v1.0 sono stati ricevuti dei consigli circa la possibilità di utilizzo di Prisma come ORM (Object-relational mapping), un sistema intermedio tra database e NextJs che facilita le query al database, anche in caso di cambio di quest'ultimo. \\Sempre a tal proposito, per usare Prisma vi sarebbe la possibilità di cambiare il gestore del database da SQLite3 a Postgres, un altro database relazionale. \\Il gruppo esaminerà la utilità di questa proposta e prenderà le necessarie decisioni.
    \item Si è stabilito di svolgere incontri in videochiamata con cadenza bimensile, sempre con la possibilità di effettuare contatti asincroni via Slack per aggiornamenti ricorrenti.
    
    \item Infine è stato ripresentato il Proof of Concept, già presentato e accettato nel precedente incontro, in forma completa e finale. Il PoC esposto presentava le funzionalità di chat già presentate in riunioni precedenti, con un'interfaccia utente migliorata, la cronologia della chat, l'aggiunta della pagina del caricamento dei documenti e il sistema di gestione di documenti mediante MinIO, come suggerito precedentemente dai proponenti.\\
    Il PoC ha sortito un esito positivo da parte dei proponenti.
\end{enumerate}

\subsection{Prossima riunione} \label{subsec:riunione}
È stata fissata una riunione per mercoledì 14 febbraio, nella quale si mostreranno i primi avanzamenti in ottica MVP.
