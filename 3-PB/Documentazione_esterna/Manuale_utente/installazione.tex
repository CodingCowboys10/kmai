\chapter{Installazione} \label{cap:Inst}
\section{Clonazione Repository da Github}
L'applicazione KMAI può essere scaricata gratuitamente da Github al seguente link: \url{https://github.com/CodingCowboys10/kmai} \textit{(Ultimo accesso: 2024/04/04)}; \\
Oppure clonata mediante terminale con il comando:
    \begin{lstlisting}
        $ git clone https://github.com/CodingCowboys10/kmai.git
    \end{lstlisting}

\section{Configurazione}
Recarsi nella cartella appena copiata alla seguente directory: \\
kmai/3-PB/mvp e aprire il file .env.local\\ \textit{(se il file non fosse visibile attivare la visualizzazione dei file nascosti)}\\
Nella prima riga, OPENAI\_API\_KEY inserire tra i doppi apici la key di OpenAI. es. sk-xxxxxx...\\
Nella seconda riga NEXT\_PUBLIC\_HOSTED\_VERSION impostare:\\ true se si desidera utilizzare la versione WEB\_only,\\ false per le modalità FULL.\\
Salvare e chiudere il file.

\section{Docker}
L'applicazione per essere avviata necessita dell'installazione di Docker, come riportato nella sezione \ref{cap:Req}.\\
All'interno della directory: kmai/3-PB/mvp eseguire da terminale il comando di installazione della versione desiderata: \\
versione GPU
    \begin{lstlisting}
        $ sudo docker compose -f docker-compose-gpu.yml up
    \end{lstlisting}
versione CPU:
    \begin{lstlisting}
        $ sudo docker compose -f docker-compose-cpu.yml up
    \end{lstlisting}
Attendere la fine dell'installazione ed infine accedere all'indirizzo web: \\\\
\url{http://localhost:3000}
