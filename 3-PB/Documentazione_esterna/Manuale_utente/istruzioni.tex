\chapter{Utilizzo del prodotto}\label{chap:istruzioni}
\section{Documenti}
\subsection{Aggiunta documento}
Per aggiungere un documento, cliccare sul pulsante "Aggiungi documento" presente nel menù a sinsitra.
\begin{figure}[h!]
    \centering
    \includegraphics[width=0.8\textwidth]{schermatadocaggiungi.png}
    \caption{Aggiungi documento nel menù}\label{fig:adddocs}
\end{figure}
\\Si aprirà una finestra da cui si potrà selezionare il file da caricare trascinandolo o scegliendolo dai file in sistema.
\begin{figure}[h!]
    \centering
    \includegraphics[width=0.6\textwidth]{dialogadd.png}
    \caption{Dialog aggiunta documento}\label{fig:dialogadd}
\end{figure}
\\Infine, cliccando su "Aggiungi documento", il documento verrà aggiunto alla lista dei documenti consultabili con il modello selezionato.
\begin{figure}[h!]
    \centering
    \includegraphics[width=0.6\textwidth]{docadd.png}
    \caption{Conferma aggiunta documento}\label{fig:confirmadd}
\end{figure}
\\Importante ricordare che i documenti caricati usando un modello locale non possono essere interrogati con il chatbot che utilizza OpenAI e viceversa.
\subsection{Ricerca documenti}
Nella parte superiore della sezione documenti è presente una barra di ricerca, in cui è possibile cercare documenti per nome o per data.
\begin{figure}[h!]
    \centering
    \includegraphics[width=0.75\textwidth]{schermatadocsearch.png}
    \caption{Barra ricerca documenti}\label{fig:searchdocs}
\end{figure}
\\Cliccando sul pulsante accanto alla barra di ricerca, è possibile scegliere il criterio con cui effettuare la ricerca.
\begin{figure}[h!]
    \centering
    \includegraphics[width=0.56\textwidth]{cercadocnome.png}
    \caption{Cerca documento per nome}\label{fig:searchdocsname}
\end{figure}
\begin{figure}[h!]
    \centering
    \includegraphics[width=0.56\textwidth]{cercadocdata.png}
    \caption{Cerca documento per data}\label{fig:searchdocsdate}
\end{figure}
\subsection{Impostazioni documento}
Cliccando sul pulsante a tre puntini accanto al documento, si aprirà un menù da cui è possibile scegliere tre diverse opzioni.
\begin{figure}[h!]
    \centering
    \includegraphics[width=0.72\textwidth]{schermatadoctrepunti.png}
    \caption{Menù documento}\label{fig:menudocs}
\end{figure}
\subsubsection{Visualizzazione documento}
Cliccando su "Visualizza" sarà possibile visualizzare il documento caricato. In caso si tratti di un pdf verrà aperto in un'altra scheda, altrimenti, se il file caricato è di un qualsiasi altro formato supportato dall'applicazione, verrà scaricato.  
\begin{figure}[h!]
    \centering
    \includegraphics[width=0.16\textwidth]{menuactiondocvisualizza.png}
    \caption{Visualizza documento}\label{fig:seedocs}
\end{figure}
\subsubsection{Modifica visibilità documento}
Cliccando su "Cambia visibilità" il documento verrà reso visibile o meno. Importante ricordare che è possibile effettuare domande solo su documenti visibili e che un documento non visibile non è eliminato dal database, ma solo nascosto.
\begin{figure}[h!]
    \centering
    \includegraphics[width=0.16\textwidth]{menuactiondocvisibility.png}
    \caption{Modifica visibilità documento}\label{fig:visibilitydocs}
\end{figure}
\subsubsection{Eliminazione documento}
Cliccando su "Elimina" il documento verrà eliminato dal database e non sarà più possibile fare domande sul contenuto di quel documento.
\begin{figure}[h!]
    \centering
    \includegraphics[width=0.16\textwidth]{menuactiondocdelete.png}
    \caption{Elimina documento}\label{fig:deletedocs}
\end{figure}
\\Per evitare eliminazioni accidentali, verrà richiesta una conferma prima di procedere.
\begin{figure}[h!]
    \centering
    \includegraphics[width=0.6\textwidth]{confirmdeletedoc.png}
    \caption{Conferma eliminazione documento}\label{fig:dialogdelete}
\end{figure}   
\subsection{Impostazioni tabella}
\subsubsection{Colonne}
Cliccando sul pulsante "Colonne" sopra la tabella si aprirà un menù a tendina. 
\begin{figure}[h!]
    \centering
    \includegraphics[width=0.8\textwidth]{schermatadocoloumns.png}
    \caption{Impostazioni tabella}\label{fig:settingtable}
\end{figure}
\\Da lì è possibile scegliere quali colonne visualizzare nella tabella dei documenti.
\begin{figure}[h!]
    \centering
    \includegraphics[width=0.25\textwidth]{visualizzacoldoc.png}
    \caption{Colonne tabella}\label{fig:coltable}
\end{figure}
\subsubsection{Righe}
È possibile scegliere di ordinare le righe della tabella in base a nome o data cliccando sulle frecce a lato dei titoletti. Inoltre, è possibile scegliere quante righe visualizzare e, grazie alle frecce in basso a destra, è possible navigare tra le eventuali pagine.
\begin{figure}[h!]
    \centering
    \includegraphics[width=0.8\textwidth]{visualizzaorderdoc.png}
    \caption{Righe tabella}\label{fig:rowtable}
\end{figure}
\subsection{Impostazioni applicazione}
Cliccando sulla rotellina presente nel menù di sinistra si apriranno le impostazioni dell'applicazione.
\begin{figure}[h!]
    \centering
    \includegraphics[width=0.8\textwidth]{schermatadocsetting.png}
    \caption{Impostazioni dell'applicazione}\label{fig:settingdoc}
\end{figure}
\subsubsection{Modifica modello}
Dalle impostazioni è possibile scegliere il modello con cui fare domande sui documenti. Importante ricordare che i documenti caricati usando un modello locale non possono
essere interrogati con il chatbot che utilizza OpenAI e viceversa.
\begin{figure}[h!]
    \centering
    \includegraphics[width=0.364\textwidth]{settingdocmodel.png}
    \caption{Cambio modello}\label{fig:settingdocmodel}
\end{figure}
\subsubsection{Modifica tema}
È possibile cambiare il tema dell'applicazione selezionando tra modalità dark (rappresentata dalla luna) e modalità light (indicata dall'icona del sole).
\begin{figure}[h!]
    \centering
    \includegraphics[width=0.364\textwidth]{settingdoctheme.png}
    \caption{Cambio tema}\label{fig:settingdoctheme}
\end{figure}
\subsubsection{Cambio pagina}
Per passare alla pagina chat basta cliccare sul pulsante "Vai alla chat" presente in fondo alle impostazioni.
\begin{figure}[h!]
    \centering
    \includegraphics[width=0.364\textwidth]{settingdocchange.png}
    \caption{Vai alla pagina chat}\label{fig:changepage}
\end{figure}

\section{Chatbot}
\subsection{}