\chapter{Requisiti}\label{sec:requisiti}
\section{Introduzione}
In questa sezione del documento vengono riportati tutti i requisiti, relativi al prodotto del progetto, individuati a seguito della fase di analisi.\\
Ogni requisito sarà identificato da un codice alfanumerico, che andrà ad identificare tipologia (funzionale, di qualità, di vincolo, prestazionale o implementativo), classificazione (obbligatorio, desiderabile, opzionale) e numero del requisito.

\section{Requisiti funzionali}

\begingroup
\setlength{\tabcolsep}{10pt}
\renewcommand{\arraystretch}{1.5}
\rowcolors{2}{oddrow}{evenrow}
\begin{xltabular}{\textwidth}{| c | X | c | c |}
    \hline
    \rowcolor{headerrow} \textbf{\textcolor{white}{Codice}} & \textbf{\textcolor{white}{Descrizione}} & \textbf{\textcolor{white}{Classificazione}} & \textbf{\textcolor{white}{Fonte}}\\
    \hline
    \endhead
    RFO-1 & L’utente deve poter visualizzare la lista di tutti i modelli LLM supportati dal sistema. & Obbligatorio & UC-1 \\
    \hline
    RFO-2 & L’utente deve poter selezionare il LLM che il sistema deve utilizzare per le operazioni sui documenti e per la generazione delle risposte. & Obbligatorio & UC-2 \\
    \hline
    RFO-3 & L’utente deve poter visualizzare la lista dei documenti presenti nel sistema e processati, sui quali poi potrà fare domande tramite chatbot. & Obbligatorio & UC-3 \\
    \hline
    RFO-4 & L'utente deve poter visualizzare delle informazioni per ciascun documento presente nella lista & Obbligatorio & UC-4 \\
    \hline
    RFO-5 & L'utente deve poter visualizzare il nome del documento di interesse. & Obbligatorio & UC-4.1 \\
    \hline
    RFO-6 & L’utente deve poter visualizzare la data di inserimento del documento di interesse. & Obbligatorio & UC-4.2 \\
    \hline
    RFZ-7 & L’utente deve poter visualizzare i tag applicati al documento di interesse. I tag sono etichette associate ad uno o più documenti che permettono la loro classificazione. & Opzionale & UC-4.3 \\
    \hline
    RFO-8 & L’utente deve poter visualizzare il contenuto del documento di interesse. & Obbligatorio & UC-4.4 \\
    \hline
    RFD-9 & L’utente deve poter visualizzare lo stato del documento di interesse (bloccato o non bloccato), così da sapere se il chatbot è abilitato a fornire risposte su tale documento. & Desiderabile & UC-4.5 \\
    \hline
    RFO-10 & L’utente deve poter visualizzare la dimensione del documento di interesse. & Obbligatorio & UC-4.6 \\
    \hline
    RFO-11 & L’utente deve poter effettuare una ricerca dei documenti & Obbligatorio & UC-5 \\
    \hline
    RFO-12 & L’utente deve poter ricercare un documento per nome. & Obbligatorio & UC-5.1 \\
    \hline
    RFO-13 & L’utente deve poter ricercare un documento per data di inserimento nel sistema. & Obbligatorio & UC-5.2 \\
    \hline
    RFZ-14 & L’utente deve poter ricercare un documento per i propri tag. & Opzionale & UC-5.3 \\
    \hline
    RFZ-15 & L’utente deve poter aggiungere un tag ad un documento. & Opzionale & UC-6 \\
    \hline
    RFZ-16 & L’utente deve poter rimuovere un tag da un documento a cui è associato. & Opzionale & UC-7 \\
    \hline
    RFZ-17 & L’utente deve poter creare un nuovo tag da salvare nel sistema. & Opzionale & UC-8 \\
    \hline
    RFZ-18 & L’utente deve poter aggiungere un nome al tag durante la sua creazione. & Opzionale & UC-8.1 \\
    \hline
    RFZ-19 & L’utente deve poter aggiungere un colore al tag durante la sua creazione. & Opzionale & UC-8.2 \\
    \hline
    RFZ-20 & L’utente deve poter aggiungere una descrizione al tag durante la sua creazione. & Opzionale & UC-8.3 \\
    \hline
    RFZ-21 & L’utente deve poter visualizzare la lista di tutti i tag presenti nel sistema e associabili a un documento. & Opzionale & UC-9 \\
    \hline
    RFZ-22 & L’utente deve poter visualizzare delle informazioni su ciascun tag presente nella lista & Opzionale & UC-10 \\
    \hline
    RFZ-23 & L’utente deve poter visualizzare il nome di ogni tag presente nel sistema. & Opzionale & UC-10.1 \\
    \hline
    RFZ-24 & L’utente deve poter visualizzare il colore di ogni tag presente nel sistema. & Opzionale & UC-10.2 \\
    \hline
    RFZ-25 & L’utente deve poter visualizzare la descrizione di ogni tag presente nel sistema. & Opzionale & UC-10.3 \\
    \hline
    RFZ-26 & L’utente deve poter eliminare uno dei tag presenti nel sistema e con esso tutte le associazioni a ogni documento. & Opzionale & UC-11 \\
    \hline
    RFO-27 & L’utente deve poter eliminare uno dei documenti presenti nel sistema e con esso tutte le informazioni ad esso associate. & Obbligatorio & UC-12 \\
    \hline
    RFO-28 & L’utente deve poter confermare l’eliminazione di uno dei documenti presenti nel sistema e solo dopo la conferma può avvenire l'eliminazione definitiva di ogni informazione associata a quel documento. & Obbligatorio & UC-12.1 \\
    \hline
    RFO-29 & L’utente deve poter aggiungere dei documenti nel sistema. & Obbligatorio & UC-13 \\
    \hline
    RFD-30 & L’utente deve poter aggiungere un documento nel sistema tramite trascinamento (drag and drop). & Desiderabile & UC-13.1 \\
    \hline
    RFO-31 & L’utente deve poter aggiungere un documento nel sistema tramite navigazione del file system. & Obbligatorio & UC-13.2 \\
    \hline
    RFD-32 & L'utente deve poter visualizzare un messaggio che lo informa che non è stato possibile inserire il documento. & Desiderabile  & UC-14 \\
    \hline
    RFD-33 & L'utente deve poter visualizzare un messaggio che lo informa che non è stato possibile inserire il documento a causa del nome del file già in uso. & Desiderabile & UC-14.1 \\
    \hline
    RFD-34 & L'utente deve poter visualizzare un messaggio che lo informa che non è stato possibile inserire il documento a causa del formato del file non supportato. & Desiderabile & UC-14.2 \\
    \hline
    RFD-35 & L'utente deve poter visualizzare un messaggio che lo informa che non è stato possibile inserire il documento a causa della corruzione del file. & Desiderabile & UC-14.3 \\
    \hline
    RFD-36 & L’utente deve poter bloccare un documento sbloccato, così che il sistema non possa fornire risposte su quel particolare documento. & Desiderabile & UC-15 \\
    \hline
    RFD-37 & L’utente deve poter sbloccare un documento in precedenza bloccato, così che il sistema possa nuovamente fornire risposte su quel particolare documento. & Desiderabile & UC-16 \\
    \hline
    RFD-38 & L’utente deve poter visualizzare la lista delle lingue supportate dal chatbot. & Desiderabile & UC-17 \\
    \hline
    RFD-39 & L’utente deve poter selezionare la lingua utilizzata dal sistema nel fornire le risposte alle sue domande. & Desiderabile & UC-18 \\
    \hline
    RFO-40 & L’utente deve poter inviare le domande da porgere al chatbot. & Obbligatorio & UC-19 \\
    \hline
    RFO-41 & L’utente deve poter digitare la domanda da porgere al chatbot tramite tastiera. & Obbligatorio & UC-19.1 \\
    \hline
    RFD-42 & L’utente deve poter inserire la domanda da porgere al chatbot tramite microfono. & Desiderabile & UC-19.2 \\
    \hline
    RFD-43 & Il sistema, dopo non aver registrato alcun input vocale nel tempo limite a seguito del tentativo da parte dell'utente di inserire una domanda tramite microfono, deve notificare un messaggio all'utente che avvisa la mancata trascrizione della domanda. & Desiderabile & UC-20 \\
    \hline
    RFO-44 & L’utente deve poter visualizzare la risposta alla domanda che ha inviato in precedenza. & Obbligatorio & UC-21 \\
    \hline
    RFO-45 & L’utente deve poter visualizzare la risposta alla domanda che ha inviato in precedenza, qualora l'informazione sia contenuta all'interno di uno dei documenti presenti nel sistema. & Obbligatorio & UC-21.1 \\
    \hline
    RFO-46 & L’utente deve poter visualizzare una risposta di cortesia prodotta dal sistema dopo la ricezione di una domanda non pertinente con alcuna informazione presente in tutti i documenti. & Obbligatorio & UC-21.2 \\
    \hline
    RFO-47 & L'utente deve poter visualizzare un messaggio che lo informa che c'è stato un errore nel ricevere la risposta entro il tempo limite. & Obbligatorio & UC-22 \\
    \hline
    RFD-48 & L’utente deve poter creare una nuova sessione di conversazione col chatbot. & Desiderabile & UC-23 \\
    \hline
    RFD-49 & L’utente deve poter visualizzare la lista delle sessioni di conversazione col chatbot attive. & Desiderabile & UC-24 \\
    \hline
    RFD-50 & L’utente deve poter eliminare tutte le sessioni di conversazione attive nel sistema e con esse tutti i messaggi scambiati in quelle conversazioni. & Desiderabile & UC-25 \\
    \hline
    RFD-51 & L’utente deve poter confermare l’eliminazione di tutte le sessioni di conversazione attive nel sistema e con esse tutti i messaggi scambiati in quelle conversazioni. & Desiderabile & UC-25.1 \\
    \hline
    RFO-52 & L’utente deve poter visualizzare lo scambio di domande e risposte avvenuto in precedenza con il sistema in una stessa sessione. & Obbligatorio & UC-26 \\
    \hline
    RFO-53 & L’utente deve poter eliminare una delle sessioni di conversazioni attive nel sistema e con essa tutti i messaggi scambiati in quella conversazione. & Obbligatorio & UC-27 \\
    \hline
    RFO-54 & L’utente deve poter confermare l’eliminazione di una sessione di conversazione e solo dopo deve avvenire l'eliminazione effettiva dei dati associati. & Obbligatorio & UC-27.1 \\
    \hline
    RFO-55 & L’utente deve poter visualizzare le fonti relative alla risposta. & Obbligatorio & UC-28 \\
    \hline
    RFO-56 & L’utente deve poter visualizzare il nome del documento relativo alla risposta. & Obbligatorio & UC-28.1 \\
    \hline
    RFD-57 & L'utente deve poter visualizzare il numero della pagina del documento relativo alla risposta. & Desiderabile & UC-28.2 \\
    \hline
    RFD-58 & L’utente deve poter sentire la lettura della risposta ricevuta. & Desiderabile & UC-29 \\
    \hline
    RFO-59 & Il sistema deve garantire la creazione di almeno una sessione di conversazione. & Obbligatorio & Proponente\\ %funzionale
    \hline
    RFD-60 & Il sistema deve garantire la creazione di almeno due sessioni di conversazione. & Desiderabile & Interna\\ %funzionale
    \hline
    RFO-61 & Il sistema deve interrompere in modo automatico il processo di generazione della risposta ad una domanda, qualora esso dovesse impiegare un tempo superiore ai 30 secondi. & Obbligatorio & UC-18\\
    \hline
    RFD-62 & Il sistema deve interrompere in modo automatico la trascrizione della domanda via input vocale, qualora non fosse rilevato alcuna voce per 5 secondi. & Desiderabile & UC-19.2\\
    \hline
    RFO-63 & Il sistema deve salvare, in modo persistente, i messaggi scambiati tra utente e chatbot. & Obbligatorio & UC-24\\
    \hline
    RFO-64 & Il sistema deve effettuare una ricerca semantica (semantic search) tra la domanda posta dall'utente e i gli \ccgloss{embedding} dei documenti caricati restituendo il documento, o pagina di esso, inerente alla domanda. & Obbligatorio & UC-21.1\\
    \hline
    RFO-65 & Il sistema deve interrogare il LLM scelto in base alla domanda posta dall'utente e al relativo documento. & Obbligatorio & UC-19\\
    \hline
    RFO-66 & Il sistema deve supportare il caricamento di file PDF. & Obbligatorio & Proponente \\
    \hline
    RFD-67 & Il sistema deve supportare il caricamento di file PDF/A. & Desiderabile & Interna \\
    \hline
    RFD-68 & Il sistema deve supportare il caricamento di  file con formato .docx, prodotti con Microsoft Word 2007 e versioni successive. & Desiderabile & Proponente \\
    \hline
    RFZ-69 & Il sistema deve supportare il caricamento di  file con formato .mp3. & Opzionale & Proponente, Interna \\
    \hline
    RFZ-70 & Il sistema deve supportare il caricamento di  file con formato .mp4. & Opzionale & Proponente, Interna \\
    \hline
    
    
  
    \rowcolor{white} \caption{Requisiti funzionali del prodotto}
    \label{tab:reqfun}
\end{xltabular}
\endgroup
  
\section{Requisiti di qualità}

\begingroup
\setlength{\tabcolsep}{10pt}
\renewcommand{\arraystretch}{1.3}
\rowcolors{2}{oddrow}{evenrow}
\begin{xltabular}{\textwidth}{| c | X | c | c |}
    \hline
    \rowcolor{headerrow} \textbf{\textcolor{white}{Codice}} & \textbf{\textcolor{white}{Descrizione}} & \textbf{\textcolor{white}{Classificazione}} & \textbf{\textcolor{white}{Fonte}}\\
    \hline
    \endhead
    RQO-1 & Deve essere fornito un documento denominato Analisi dei costi, rischi e tecnologie. & Obbligatorio & Proponente \\
    \hline
    RQO-2 & Deve essere fornito un documento che descrive le attività di \ccgloss{bug} reporting effettuate. & Obbligatorio & Capitolato \\
    \hline
    RQO-3 & Il progetto deve essere svolto seguendo le regole stabilite dal documento Norme\_di\_progetto. & Obbligatorio & Interna \\
    \hline
    RQO-4 & Deve essere fornito al proponente il codice sorgente in un \ccgloss{repository} GitHub. & Obbligatorio & Proponente \\
    \hline
    RQO-5 & Deve essere fornito il documento Manuale\_utente. & Obbligatorio & Regolamento \\
    \hline
    RQO-6 & Deve essere fornito il documento Specifica\_architetturale. & Obbligatorio & Regolamento \\
    \hline
    \rowcolor{white} \caption{Requisiti di qualità del prodotto}
    \label{tab:reqqua}
\end{xltabular}
\endgroup
\newpage
\section{Requisiti di vincolo}

\begingroup
\setlength{\tabcolsep}{10pt}
\renewcommand{\arraystretch}{1.5}
\rowcolors{2}{oddrow}{evenrow}
\begin{xltabular}{\textwidth}{| c | X | c | c |}
    \hline
    \rowcolor{headerrow} \textbf{\textcolor{white}{Codice}} & \textbf{\textcolor{white}{Descrizione}} & \textbf{\textcolor{white}{Classificazione}} & \textbf{\textcolor{white}{Fonte}}\\
    \hline
    \endhead
   
    RVO-1 & Il sistema deve permettere l'utilizzo di LLM tramite \ccgloss{OpenAI}. & Obbligatorio & Proponente, Interna \\
    \hline
    RVD-2 & Il sistema deve permettere il riconoscimento vocale tramite \ccgloss{Whisper} di OpenAI. & Desiderabile & Proponente, Interna \\
    \hline
    RVO-3 & Il sistema deve permettere l'utilizzo di LLM locali tramite \ccgloss{Ollama} 0.1.19 o successivo come garanzia di sicurezza per tutti i dati contenuti nel sistema. & Obbligatorio & Proponente, Interna \\
    \hline
    RVO-4 & La web application deve essere sviluppata in \ccgloss{React} 18.2.0 o successivo. & Obbligatorio & Proponente, Interna \\
    \hline
    RVO-5 & Dev'essere utilizzato il \ccgloss{framework} \ccgloss{Langchain} 0.0.208 o successivo per il collegamento al \ccgloss{database} vettoriale e al LLM. & Obbligatorio & Proponente, Interna \\
    \hline
    RVO-6 & Dev'essere utilizzato il database vettoriale \ccgloss{ChromaDB} 1.6.1 o successivo per la persistenza dei vettori embedding dei documenti. & Obbligatorio & Proponente, Interna \\
    \hline
    RVO-7 & Dev'essere utilizzato \ccgloss{Nodejs} 20.11.0 o successivo per lo sviluppo del \ccgloss{back-end}. & Obbligatorio & Proponente, Interna \\
    \hline
    RVO-8 & Dev'essere utilizzato il framework \ccgloss{Nextjs} versione 14.0.3 o successivo per lo sviluppo del back-end. & Obbligatorio & Proponente, Interna \\
    \hline
    RVO-9 & Dev'essere utilizzato \ccgloss{Postgres} 8.11.3 o successivo per il salvataggio delle sessioni e dei messaggi scambiati tra utente e chatbot. & Obbligatorio & Interna \\
    \hline
    RVO-10 & Dev'essere utilizzato il framework \ccgloss{Tailwind} 3.3.0 o successivo per codificare lo stile del \ccgloss{front-end}. & Obbligatorio & Interna \\
    \hline
    RVO-11 & Dev'essere utilizzato \ccgloss{MinIO} 8.4.3 o successivo per l'archiviazione dei documenti. & Obbligatorio & Proponente, Interna \\
    \hline
    RVO-12 & Dev'essere utilizzato \ccgloss{Docker} 24.0.7 o successivo per l'esecuzione del prodotto. & Obbligatorio & Interna \\
    \hline
    RVO-13 & Il sistema deve garantire il suo funzionamento con la presenza a sistema di 1000 documenti, tra i formati supportati.  & Obbligatorio & Proponente\\ %vincolo
    \hline
    RVO-14 & Il sistema deve garantire il suo funzionamento con documenti, tra i formati supportati, di dimensione fino a 500KB. & Obbligatorio & Proponente\\ %vincolo
    \hline
    RVZ-15 & Il sistema deve garantire il suo funzionamento con audio, tra i formati supportati, di dimensione fino a 5MB. & Opzionale & Proponente\\
    \hline % vincolo
    RVO-16 & Il sistema deve essere utilizzabile correttamente nel browser Google Chrome dalla versione 110 e successive. & Obbligatorio & Proponente, Interna\\  % vincolo
    \hline
    RVO-17 & Il sistema deve essere utilizzabile correttamente nel browser Mozilla Firefox dalla versione 116 e successive. & Obbligatorio & Proponente, Interna\\  % vincolo
    \hline
    RVO-18 & Il sistema deve essere utilizzabile correttamente nel browser Opera dalla versione 96 e successive. & Obbligatorio & Proponente, Interna\\ % vincolo
    \hline
    RVO-19 & Il sistema deve essere utilizzabile correttamente nel browser Microsoft Edge dalla versione 110 e successive. & Obbligatorio & Proponente, Interna\\
    \hline
    \rowcolor{white} \caption{Requisiti di vincolo del prodotto}
    \label{tab:reqvin}
\end{xltabular}
\endgroup

\section{Requisiti prestazionali}
Per poter utilizzare un Large Language Model fornito da OpenAI è necessario il possesso di una chiave \ccgloss{API} valida, oltre ad una connessione ad Internet.\\
Per l'utilizzo di Ollama e di MinIO sono necessarie le loro installazioni ed esecuzioni tramite i docker indicati nei requisiti di installazione del prodotto.
Nell'utilizzo di un LLM locale, le prestazioni dell'applicazione sono altamente influenzate dal sistema che la esegue. Pertanto, per poter soddisfare i requisiti funzionali e di vincolo individuati, è necessario l'utilizzo di un sistema dotato non meno di 16GB di RAM.\\
Ogni sistema che non soddisfa questi requisiti, è quindi da ritenersi improprio all'utilizzo dell'applicazione tramite tecnologie in locale, e non può essere garantito il corretto funzionamento del prodotto, né il completo rispetto dei requisiti. Inoltre, le tecnologie individuate per l'utilizzo di LLM locali richiedono l'adozione di un sistema operativo Linux o MacOS 11 o superiori.

\section{Requisiti implementativi}
\begingroup
\setlength{\tabcolsep}{10pt}
\renewcommand{\arraystretch}{1.5}
\rowcolors{2}{oddrow}{evenrow}
\begin{xltabular}{\textwidth}{| c | X | c | c |}
    \hline
    \rowcolor{headerrow} \textbf{\textcolor{white}{Codice}} & \textbf{\textcolor{white}{Descrizione}} & \textbf{\textcolor{white}{Classificazione}} & \textbf{\textcolor{white}{Fonte}}\\
    \hline
    \endhead
    RIO-1 & Il sistema deve processare i documenti che vengono caricati, creandone i loro embedding. & Obbligatorio & Interna\\
    \hline
    RIO-2 & Il sistema deve salvare, in modo persistente, i documenti caricati a sistema. & Obbligatorio & Proponente\\
    \hline
    RIO-3 & Il sistema deve salvare, in modo persistente, tutte le informazioni relative ai documenti presenti a sistema. & Obbligatorio & Interna\\
    \hline
    RIO-4 & Il sistema deve salvare, in modo persistente, i vettori dei documenti caricati ed embeddizzati dal sistema. & Obbligatorio & Proponente\\
    \hline
    
    \rowcolor{white} \caption{Requisiti implementativi del prodotto}
    \label{tab:reqimp}
\end{xltabular}
\endgroup
\newpage
\section{Tracciamento requisiti}

\subsection{Fonte-Requisiti}
\renewcommand{\arraystretch}{1.45}
\rowcolors{2}{oddrow}{evenrow}
\begin{xltabular}{\textwidth}{|X| X |}
    \hline
    \rowcolor{headerrow} \textbf{\textcolor{white}{Fonte}} & \textbf{\textcolor{white}{Requisiti}}\\
    \hline
    \endhead
    Proponente & RFO-59, \newline
                RFO-66, \newline
                RFD-68, \newline
                RFZ-69, \newline
                RFZ-70, \newline
                RQO-1, \newline
                RQO-4, \newline
                RVO-1, \newline
                RVD-2, \newline
                RVO-3, \newline
                RVO-4, \newline
                RVO-5, \newline
                RVO-6, \newline
                RVO-7, \newline
                RVO-8, \newline
                RVO-11, \newline
                RVO-13, \newline
                RVO-14, \newline
                RVZ-15, \newline
                RVO-16, \newline
                RVO-17, \newline
                RVO-18, \newline
                RVO-19, \newline
                RIO-2, \newline
                RIO-4 \\
    \hline
     Interna &  RFD-60, \newline 
                RFO-67, \newline
                RFZ-69, \newline
                RFZ-70, \newline
                RVO-1, \newline 
                RVD-2, \newline
                RVO-3, \newline
                RVO-4, \newline
                RVO-5, \newline
                RVO-6, \newline
                RVO-7, \newline
                RVO-8, \newline
                RVO-9, \newline
                RVO-10, \newline
                RVO-11, \newline
                RVO-12, \newline
                RVO-17, \newline
                RVO-18, \newline
                RVO-19, \newline
                RQO-3, \newline
                RIO-1, \newline
                RIO-3 \\
    \hline
    Capitolato & RQO-2 \\
    \hline
    Regolamento & RQO-5, \newline 
                 RQO-6 \\
    \hline
    UC-1 & RFO-1  \\
    \hline
    UC-2 & RFO-2  \\
    \hline
    UC-3 & RFO-3  \\
    \hline
    UC-4 & RFO-4 \\
    \hline
    UC-4.1 & RFO-5\\
    \hline
    UC-4.2 & RFO-6\\
    \hline
    UC-4.3 & RFZ-7 \\
    \hline
    UC-4.4 & RFO-8\\ 
    \hline
    UC-4.5 & RFD-9\\
    \hline
    UC-4.6 & RFO-10\\
    \hline
    UC-5 & RFO-11\\
    \hline
    UC-5.1 & RFO-12\\
    \hline
    UC-5.2 & RFO-13\\
    \hline
    UC-5.3 & RFZ-14 \\
    \hline
    UC-6 & RFZ-15\\
    \hline
    UC-7 & RFZ-16 \\
    \hline
    UC-8 & RFZ-17\\
    \hline
    UC-8.1 & RFZ-18\\
    \hline
    UC-8.2 & RFZ-19\\
    \hline
    UC-8.3 & RFZ-20\\
    \hline
    UC-9 & RFZ-21\\
    \hline
    UC-10 & RFZ-22\\
    \hline
    UC-10.1 & RFZ-23 \\
    \hline
    UC-10.2 & RFZ-24 \\
    \hline
    UC-10.3 & RFZ-25 \\
    \hline
    UC-11 & RFZ-26 \\
    \hline
    UC-12 & RFO-27\\
    \hline
    UC-12.1 & RFO-28\\
    \hline
    UC-13 & RFO-29\\
    \hline
    UC-13.1 & RFO-30 \\
    \hline
    UC-13.2 & RFD-31\\
    \hline
    UC-14 & RFO-32 \\
    \hline
    UC-14.1 & RFD-33  \\
    \hline
    UC-14.2 & RFD-34\\
    \hline
    UC-14.3 & RFD-35\\
    \hline
    UC-15 & RFD-36 \\
    \hline
    UC-16 & RFD-37\\
    \hline
    UC-17 & RFD-38 \\
    \hline
    UC-18 & RFD-39,\newline RFO-61  \\
    \hline
    UC-19 & RFO-40, \newline RFO-65 \\
    \hline
    UC-19.1 & RFO-41, \\
    \hline
    UC-19.2 & RFD-42,\newline RFD-62 \\
    \hline
    UC-20 & RFD-43 \\
    \hline
    UC-21 & RFO-44 \\
    \hline
    UC-21.1 & RFO-45, \newline RFO-64 \\
    \hline
    UC-21.2 & RFO-46 \\
    \hline
    UC-22 & RFO-47  \\
    \hline
    UC-23 & RFD-48 \\
    \hline
    UC-24 & RFD-49, \newline RFO-63 \\
    \hline
    UC-25 & RFD-50  \\
    \hline
    UC-25.1 & RFD-51  \\
    \hline
    UC-26 & RFO-52 \\
    \hline
    UC-27 & RFO-53  \\
    \hline
    UC-27.1 & RFO-54 \\
    \hline
    UC-28 & RFO-55  \\
    \hline
    UC-28.1 & RFO-56 \\
    \hline
    UC-28.2 & RFD-57 \\
    \hline
    UC-29 & RFD-58 \\
   \hline
    \rowcolor{white} \caption{Tracciamento fonte-requisiti}
    \label{tab:riepilogo}
\end{xltabular}

\subsection{Requisiti-Fonte}
\renewcommand{\arraystretch}{1.45}
\rowcolors{2}{oddrow}{evenrow}
\begin{xltabular}{\textwidth}{|X | X|}
    \hline
    \rowcolor{headerrow} \textbf{\textcolor{white}{Requisiti}} & \textbf{\textcolor{white}{Fonte}}\\
    \hline
    \endhead
    
    \makecell[l]{RFO-59,\\
    RFO-66,\\
    RFD-68,\\
    RFZ-69,\\
    RFZ-70,\\ RQO-1,\\ RQO-4,\\ RVO-1,\\ RVD-2,\\ RVO-3,\\ RVO-4,\\ RVO-5,
    \\ RVO-6,\\ RVO-7,\\ RVO-8,\\ RVO-11,\\ RVO-13,\\ RVO-14,
    \\ RVZ-15,\\ RVO-16,\\ RVO-17,\\ RVO-18,\\ RVO-19,\\ RIO-2,\\ RIO-4} & Proponente \\

    \hline

    \makecell[l]{RFD-60, \\ 
                RFD-67, \\
                RFZ-69, \\
                RFZ-70, \\
                RVO-1, \\ 
                RVD-2, \\
                RVO-3, \\
                RVO-4, \\
                RVO-5, \\
                RVO-6, \\
                RVO-7, \\
                RVO-8, \\
                RVO-9, \\
                RVO-10, \\
                RVO-11, \\
                RVO-12, \\
                RVO-17, \\
                RVO-18, \\
                RVO-19, \\
                RQO-3, \\
                RIO-1, \\
                RIO-3} & Interna \\
    \hline
    \makecell[l]{
    RQO-2} & Capitolato \\
    \hline
    \makecell[l]{
    RQO-5,\\
    RQO-6} & Regolamento \\
    \hline
    RFO-1 & UC-1  \\
    \hline
    RFO-2 & UC-2  \\
    \hline
    RFO-3 & UC-3 \\
    \hline
    RFO-4 & UC-4 \\
    \hline
    RFO-5 & UC-4.1 \\
    \hline
    RFO-6 & UC-4.2 \\
    \hline
    RFZ-7 & UC-4.3 \\
    \hline
    RFO-8 & UC-4.4 \\ 
    \hline
    RFD-9 & UC-4.5 \\
    \hline
    RFO-10 & UC-4.6\\
    \hline
    RFO-11 & UC-5 \\
    \hline
    RFO-12 & UC-5.1 \\
    \hline
    RFO-13 & UC-5.2 \\
    \hline
    RFZ-14 & UC-5.3 \\
    \hline
    RFZ-15 & UC-6 \\
    \hline
    RFZ-16 & UC-7 \\
    \hline
    RFZ-17 & UC-8 \\
    \hline
    RFZ-18 & UC-8.1 \\
    \hline
    RFZ-19 & UC-8.2 \\
    \hline
    RFZ-20 & UC-8.3 \\
    \hline
    RFZ-21 & UC-9 \\
    \hline
    RFZ-22 & UC-10 \\
    \hline
    RFZ-23 & UC-10.1 \\
    \hline
    RFZ-24 & UC-10.2 \\
    \hline
    RFZ-25 & UC-10.3 \\
    \hline
    RFZ-26 & UC-11 \\
    \hline
    RFO-27 & UC-12 \\
    \hline
    RFO-28 & UC-12.1 \\
    \hline
    RFO-29 & UC-13 \\
    \hline
    RFO-30 & UC-13.1 \\
    \hline
    RFD-31 & UC-13.2 \\
    \hline
    RFO-32 & UC-14 \\
    \hline
    RFD-33 & UC-14.1 \\
    \hline
    RFD-34 & UC-14.2 \\
    \hline
    RFD-35 & UC-14.3 \\
    \hline
    RFD-36 & UC-15 \\
    \hline
    RFD-37 & UC-16 \\
    \hline
    RFD-38 & UC-17 \\
    \hline
    RFD-39 & UC-18 \\
    \hline
    RFO-40 & UC-19 \\
    \hline
    RFO-41 & UC-19.1 \\
    \hline
    RFD-42 & UC-19.2 \\
    \hline
    RFD-43 & UC-20 \\
    \hline
    RFO-44 & UC-21 \\
    \hline
    RFO-45 & UC-21.1 \\
    \hline
    RFO-46 & UC-21.2\\
    \hline
    RFO-47 & UC-22 \\
    \hline
    RFD-48 & UC-23 \\
    \hline
    RFD-49 & UC-24 \\
    \hline
    RFD-50 & UC-25 \\
    \hline
    RFD-51 & UC-25.1 \\
    \hline
    RFO-52 & UC-26 \\
    \hline
    RFO-53 & UC-27 \\
    \hline
    RFO-54 & UC-27.1 \\
    \hline
    RFO-55 & UC-28 \\
    \hline
    RFO-56 & UC-28.1 \\
    \hline
    RFD-57 & UC-28.2 \\
    \hline
    RFD-58 & UC-29 \\
    \hline
    RFO-61 & UC-18 \\
    \hline
    RFD-62 & UC-19.2 \\
    \hline
    RFO-63 & UC-24 \\
    \hline
    RFO-64 & UC-21.1 \\
    \hline
    RFO-65 & UC-19\\
\hline
    \rowcolor{white} \caption{Tracciamento requisiti-fonte}
    \label{tab:riepilogo}
\end{xltabular}

\subsection{Riepilogo}
\begingroup
\setlength{\tabcolsep}{10pt}
\renewcommand{\arraystretch}{1.5}
\rowcolors{2}{oddrow}{evenrow}
\begin{xltabular}{\textwidth}{| X | c | c | c |}
    \hline
    \rowcolor{headerrow} \textbf{\textcolor{white}{Requisito}} & \textbf{\textcolor{white}{Obbligatorio}} & \textbf{\textcolor{white}{Desiderabile}} & \textbf{\textcolor{white}{Opzionale}}\\
    \hline
    \endhead
    Funzionale & 32 & 22 & 16 \\
    \hline
    Di qualità & 6 & - & - \\
    \hline
    Di vincolo & 17 & 1 & 1 \\
    \hline
    Prestazionali & - & - & - \\
    \hline
    Implementativi & 4 & - & - \\
    \hline
    \cellcolor{headerrow} \textbf{\textcolor{white}{Totale}} & 59 & 23 & 17 \\
    \hline
    \rowcolor{white} \caption{Riepilogo dei requisiti}
    \label{tab:riepilogo}
\end{xltabular}
\endgroup



