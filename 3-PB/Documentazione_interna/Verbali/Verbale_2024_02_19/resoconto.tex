\section{Resoconto} \label{sec:resoconto}
\subsection{Discussioni} \label{subsec:resdiscussione}
\begin{enumerate}
    \item L'incontro si apre con una discussione riguardante lo stato di avanzamento della fase di design/progettazione del MVP. Nonostante la preoccupazione iniziale, sono stati identificati numerosi design \ccgloss{pattern}, tipici della libreria \ccgloss{React}, relativi alla parte di \ccgloss{front-end} del prodotto.\\
    Il dubbio maggiore riguarda quale architettura preferire e quali pattern adottare per la gestione del codice del \ccgloss{back-end}. Come invitati dagli stessi a fare, in mattinata è stata avviata una conversazione con i \ccgloss{proponenti} sul gruppo \ccgloss{Slack} apposito, per chiedere consiglio a riguardo.\\
    In caso di mancata risposta nelle prossime 48 ore, il gruppo si muoverà comunque autonomamente.\\
    Contestualmente alla fase di progettazione, e in generale all'inizio della \ccgloss{PB}, sono già stati creati i documenti di Specifica\_architetturale e Manuale\_utente.
    \item La riunione prosegue rivolgendo l'attenzione alle modifiche da apportare \ccgloss{ai} documenti, seguendo le segnalazioni emerse dalla valutazione della revisione \ccgloss{RTB}. In particolare, le modifiche relative all'Analisi\_dei\_requisiti, al Piano\_di\_qualifica e alla struttura dei verbali esterni sono già state apportate. È invece da completare la correzione dei riferimenti ai vari documenti nelle Norme\_di\_progetto.\\
    Sono inoltre state corrette le sezioni dei documenti nelle quali vengono riportati riferimenti di \ccgloss{risorse} web. In particolare, ora ogni risorsa web riporta la data di ultimo accesso ad essa.
    \item Prima del termine dell'incontro, viene segnalato il malfunzionamento dello script relativo alla compilazione automatica del log delle modifiche di un documento. È quindi necessaria una correzione quanto prima.\\
    Tale incarico è stato subito assegnato ad uno degli \ccgloss{amministratori} di questo \ccgloss{sprint}.
\end{enumerate}

\subsection{Azioni da intraprendere}
{
    \setlength{\tabcolsep}{10pt}
            \renewcommand{\arraystretch}{1.5}
            \rowcolors{2}{oddrow}{evenrow}
            \begin{xltabular}{\textwidth}{| l | l | l | X |}
                 \hline
                 \rowcolor{headerrow}\textbf{\textcolor{white}{Codice \ccgloss{issue}}} & \textbf{\textcolor{white}{Assegnatario}} & \textbf{\textcolor{white}{Scadenza}} & \textbf{\textcolor{white}{Descrizione}} \\
                 \hline
                 CC-190 & Leonardo Lago & 2024/02/20 & Fix automazione log e frontespizio documenti.\\
                 \hline
                 CC-191 & Andrea Cecchin & 2024/02/20 & Stesura verbale riunione 2024/02/19.\\
                 \hline
                 
            \end{xltabular}
}

\subsection{Prossima riunione} \label{subsec:riunione}
Viene fissata una riunione per la settimana successiva.
