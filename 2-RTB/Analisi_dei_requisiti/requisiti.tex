\chapter{Requisiti}\label{sec:requisiti}
\section{Introduzione}
In questa sezione del documento vengono riportati tutti i requisiti, relativi al prodotto del progetto, individuati a seguito della fase di analisi.\\
Ogni requisito sarà identificato da un codice alfanumerico, che andrà ad identificare tipologia (funzionale, di qualità, di vincolo, prestazionale o implementativo), classificazione (obbligatorio, desiderabile, opzionale) e numero del requisito.

\section{Requisiti funzionali}

\begingroup
\setlength{\tabcolsep}{10pt}
\renewcommand{\arraystretch}{1.5}
\rowcolors{2}{oddrow}{evenrow}
\begin{xltabular}{\textwidth}{| c | X | c | c |}
    \hline
    \rowcolor{headerrow} \textbf{\textcolor{white}{Codice}} & \textbf{\textcolor{white}{Descrizione}} & \textbf{\textcolor{white}{Classificazione}} & \textbf{\textcolor{white}{Fonte}}\\
    \hline
    \endhead
    RFO-1 & L’utente deve poter visualizzare la lista di tutti i modelli LLM supportati dal sistema. & Obbligatorio & UC-1 \\
    \hline
    RFO-2 & L’utente deve poter selezionare il LLM che il sistema deve utilizzare per le operazioni sui documenti e per la generazione delle risposte. & Obbligatorio & UC-2 \\
    \hline
    RFO-3 & L’utente deve poter visualizzare la lista dei documenti presenti nel sistema e processati, sui quali poi potrà fare domande tramite chatbot. & Obbligatorio & UC-3 \\
    \hline
    RFO-4 & L'utente deve poter visualizzare il nome del documento di interesse. & Obbligatorio & UC-3.1.1 \\
    \hline
    RFO-5 & L’utente deve poter visualizzare la data di inserimento del documento di interesse. & Obbligatorio & UC-3.1.2 \\
    \hline
    RFZ-6 & L’utente deve poter visualizzare i tag applicati al documento di interesse. I tag sono etichette associate ad uno o più documenti che permettono la loro classificazione. & Opzionale & UC-3.1.3 \\
    \hline
    RFO-7 & L’utente deve poter visualizzare il contenuto del documento di interesse. & Obbligatorio & UC-3.1.4 \\
    \hline
    RFD-8 & L’utente deve poter visualizzare lo stato del documento di interesse (bloccato o non bloccato), così da sapere se il chatbot è abilitato a fornire risposte su tale documento. & Desiderabile & UC-3.1.5 \\
    \hline
    RFO-9 & L’utente deve poter visualizzare la dimensione del documento di interesse. & Obbligatorio & UC-3.1.6 \\
    \hline
    RFO-10 & L’utente deve poter ricercare un documento per nome. & Obbligatorio & UC-4.1 \\
    \hline
    RFO-11 & L’utente deve poter ricercare un documento per data di inserimento nel sistema. & Obbligatorio & UC-4.2 \\
    \hline
    RFZ-12 & L’utente deve poter ricercare un documento per i propri tag. & Opzionale & UC-4.3 \\
    \hline
    RFZ-13 & L’utente deve poter aggiungere un tag ad un documento. & Opzionale & UC-5 \\
    \hline
    RFZ-14 & L’utente deve poter rimuovere un tag da un documento a cui è associato. & Opzionale & UC-6 \\
    \hline
    RFZ-15 & L’utente deve poter creare un nuovo tag da salvare nel sistema. & Opzionale & UC-7 \\
    \hline
    RFZ-16 & L’utente deve poter aggiungere un nome al tag durante la sua creazione. & Opzionale & UC-7.1 \\
    \hline
    RFZ-17 & L’utente deve poter aggiungere un colore al tag durante la sua creazione. & Opzionale & UC-7.2 \\
    \hline
    RFZ-18 & L’utente deve poter aggiungere una descrizione al tag durante la sua creazione. & Opzionale & UC-7.3 \\
    \hline
    RFZ-19 & L’utente deve poter visualizzare la lista di tutti i tag presenti nel sistema e associabili a un documento. & Opzionale & UC-8 \\
    \hline
    RFZ-20 & L’utente deve poter visualizzare il nome di ogni tag presente nel sistema. & Opzionale & UC-8.1.1 \\
    \hline
    RFZ-21 & L’utente deve poter visualizzare il colore di ogni tag presente nel sistema. & Opzionale & UC-8.1.2 \\
    \hline
    RFZ-22 & L’utente deve poter visualizzare la descrizione di ogni tag presente nel sistema. & Opzionale & UC-8.1.3 \\
    \hline
    RFZ-23 & L’utente deve poter eliminare uno dei tag presenti nel sistema e con esso tutte le associazioni a ogni documento. & Opzionale & UC-9 \\
    \hline
    RFO-24 & L’utente deve poter eliminare uno dei documenti presenti nel sistema e con esso tutte le informazioni ad esso associate. & Obbligatorio & UC-10 \\
    \hline
    RFO-25 & L’utente deve poter confermare l’eliminazione di uno dei documenti presenti nel sistema e solo dopo la conferma può avvenire l'eliminazione definitiva di ogni informazione associata a quel documento. & Obbligatorio & UC-10.1 \\
    \hline
    RFD-26 & L’utente deve poter aggiungere un documento nel sistema tramite trascinamento (drag and drop). & Desiderabile & UC-11.1 \\
    \hline
    RFO-27 & L’utente deve poter aggiungere un documento nel sistema tramite navigazione del file system. & Obbligatorio & UC-11.2 \\
    \hline
    RFD-28 & L'utente deve poter visualizzare un messaggio che lo informa che non è stato possibile inserire il documento a causa del nome del file già in uso. & Desiderabile & UC-12.1 \\
    \hline
    RFD-29 & L'utente deve poter visualizzare un messaggio che lo informa che non è stato possibile inserire il documento a causa del formato del file non supportato. & Desiderabile & UC-12.2 \\
    \hline
    RFD-30 & L'utente deve poter visualizzare un messaggio che lo informa che non è stato possibile inserire il documento a causa della corruzione del file. & Desiderabile & UC-12.3 \\
    \hline
    RFD-31 & L’utente deve poter bloccare un documento, così che il sistema non fornisca risposte su tale documento senza doverlo eliminare. & Desiderabile & UC-13 \\
    \hline
    RFD-32 & L’utente deve poter sbloccare un documento in precedenza bloccato, così che il sistema possa nuovamente fornire risposte su quel particolare documento. & Desiderabile & UC-14 \\
    \hline
    RFD-33 & L’utente deve poter visualizzare la lista delle lingue supportate dal chatbot. & Desiderabile & UC-15 \\
    \hline
    RFD-34 & L’utente deve poter selezionare la lingua utilizzata dal sistema nel fornire le risposte alle sue domande. & Desiderabile & UC-16 \\
    \hline
    RFO-35 & L’utente deve poter digitare la domanda da porgere al chatbot tramite tastiera. & Obbligatorio & UC-17.1 \\
    \hline
    RFD-36 & L’utente deve poter inserire la domanda da porgere al chatbot tramite microfono. & Desiderabile & UC-17.2 \\
    \hline
    RFD-37 & Il sistema, dopo non aver registrato alcun input vocale nel tempo limite a seguito del tentativo da parte dell'utente di inserire una domanda tramite microfono, deve notificare un messaggio all'utente che avvisa la mancata trascrizione della domanda. & Desiderabile & UC-18 \\
    \hline
    RFO-38 & L’utente deve poter visualizzare la risposta alla domanda che ha inviato in precedenza, qualora l'informazione sia contenuta all'interno di uno dei documenti presenti nel sistema. & Obbligatorio & UC-19.1 \\
    \hline
    RFO-39 & L’utente deve poter visualizzare una risposta di cortesia prodotta dal sistema dopo la ricezione di una domanda non pertinente con alcuna informazione presente in tutti i documenti. & Obbligatorio & UC-19.2 \\
    \hline
    RFO-40 & L'utente deve poter visualizzare un messaggio che lo informa che c'è stato un errore nel ricevere la risposta entro il tempo limite. & Obbligatorio & UC-20 \\
    \hline
    RFD-41 & L’utente deve poter creare una nuova sessione di conversazione col chatbot. & Desiderabile & UC-21 \\
    \hline
    RFD-42 & L’utente deve poter visualizzare la lista delle sessioni di conversazione col chatbot attive. & Desiderabile & UC-22 \\
    \hline
    RFD-43 & L’utente deve poter eliminare una delle sessioni di conversazioni attive nel sistema e con essa tutti i messaggi scambiati in quella conversazione. & Desiderabile & UC-23 \\
    \hline
    RFD-44 & L’utente deve poter confermare l’eliminazione di una sessione di conversazione e solo dopo deve avvenire l'eliminazione effettiva dei dati associati. & Desiderabile & UC-23.1 \\
    \hline
    RFO-45 & L’utente deve poter visualizzare lo scambio di domande e risposte avvenuto in precedenza con il sistema in una stessa sessione. & Obbligatorio & UC-24 \\
    \hline
    RFO-46 & L’utente deve poter eliminare lo scambio di domande e risposte avvenuto in precedenza con il chatbot in una stessa sessione. & Obbligatorio & UC-25 \\
    \hline
    RFO-47 & L’utente deve poter confermare l'eliminazione dello scambio di domande e risposte, avvenuto in precedenza con il chatbot in una stessa sessione. & Obbligatorio & UC-25.1 \\
    \hline
    RFO-48 & L’utente deve poter visualizzare il nome del documento relativo alla risposta. & Obbligatorio & UC-26.1 \\
    \hline
    RFD-49 & L'utente deve poter visualizzare il numero della pagina del documento relativo alla risposta. & Desiderabile & UC-26.2 \\
    \hline
    RFD-50 & L’utente deve poter sentire la lettura della risposta ricevuta. & Desiderabile & UC-27 \\
    \hline
    \rowcolor{white} \caption{Requisiti funzionali del prodotto}
    \label{tab:reqfun}
\end{xltabular}
\endgroup

\section{Requisiti di qualità}

\begingroup
\setlength{\tabcolsep}{10pt}
\renewcommand{\arraystretch}{1.5}
\rowcolors{2}{oddrow}{evenrow}
\begin{xltabular}{\textwidth}{| c | X | c | c |}
    \hline
    \rowcolor{headerrow} \textbf{\textcolor{white}{Codice}} & \textbf{\textcolor{white}{Descrizione}} & \textbf{\textcolor{white}{Classificazione}} & \textbf{\textcolor{white}{Fonte}}\\
    \hline
    \endhead
    RQO-1 & Deve essere fornito un documento di analisi dei costi, rischi e tecnologie. & Obbligatorio & Proponente \\
    \hline
    RQO-2 & Deve essere fornito un documento che descrive le attività di bug reporting effettuate. & Obbligatorio & Capitolato \\
    \hline
    RQO-3 & Il progetto deve essere svolto seguendo le regole stabilite dal documento Norme\_di\_progetto. & Obbligatorio & Interna \\
    \hline
    RQO-4 & Deve essere fornito al proponente il codice sorgente in un repository GitHub. & Obbligatorio & Proponente \\
    \hline
    \rowcolor{white} \caption{Requisiti di qualità del prodotto}
    \label{tab:reqqua}
\end{xltabular}
\endgroup

\section{Requisiti di vincolo}

\begingroup
\setlength{\tabcolsep}{10pt}
\renewcommand{\arraystretch}{1.5}
\rowcolors{2}{oddrow}{evenrow}
\begin{xltabular}{\textwidth}{| c | X | c | c |}
    \hline
    \rowcolor{headerrow} \textbf{\textcolor{white}{Codice}} & \textbf{\textcolor{white}{Descrizione}} & \textbf{\textcolor{white}{Classificazione}} & \textbf{\textcolor{white}{Fonte}}\\
    \hline
    \endhead
    RVO-1 & Il sistema deve supportare il caricamento di file PDF. & Obbligatorio & Proponente \\
    \hline
    RVD-2 & Il sistema deve supportare il caricamento di file PDF/A. & Desiderabile & Interna \\
    \hline
    RVD-3 & Il sistema deve supportare il caricamento di  file con formato .docx, prodotti con Microsoft Word 2007 e versioni successive. & Desiderabile & Proponente \\
    \hline
    RVZ-4 & Il sistema deve supportare il caricamento di  file con formato .mp3. & Opzionale & Proponente, Interna \\
    \hline
    RVZ-5 & Il sistema deve supportare il caricamento di  file con formato .mp4. & Opzionale & Proponente, Interna \\
    \hline
    RVO-6 & Il sistema deve permettere l'utilizzo di LLM tramite \ccgloss{OpenAI}. & Obbligatorio & Proponente, Interna \\
    \hline
    RVD-7 & Il sistema deve permettere il riconoscimento vocale tramite \ccgloss{Whisper} di OpenAI. & Desiderabile & Proponente, Interna \\
    \hline
    RVO-8 & Il sistema deve permettere l'utilizzo di LLM locali tramite \ccgloss{Ollama} 0.1.19 come garanzia di sicurezza per tutti i dati contenuti nel sistema. & Obbligatorio & Proponente, Interna \\
    \hline
    RVO-9 & La web application deve essere sviluppata in \ccgloss{React} 18.2.0. & Obbligatorio & Proponente, Interna \\
    \hline
    RVO-10 & Dev'essere utilizzato il \ccgloss{framework} \ccgloss{Langchain} 0.0.208 per il collegamento al database vettoriale e al LLM. & Obbligatorio & Proponente, Interna \\
    \hline
    RVO-11 & Dev'essere utilizzato il database vettoriale \ccgloss{ChromaDB} 1.6.1 per la persistenza dei vettori \ccgloss{embedding} dei documenti. & Obbligatorio & Proponente, Interna \\
    \hline
    RVO-12 & Dev'essere utilizzato \ccgloss{Nodejs} 20.11.0 per lo sviluppo del back-end. & Obbligatorio & Proponente, Interna \\
    \hline
    RVO-13 & Dev'essere utilizzato il framework \ccgloss{Nextjs} versione 14.0.3 per lo sviluppo del back-end. & Obbligatorio & Proponente, Interna \\
    \hline
    RVO-14 & Dev'essere utilizzata la libreria \ccgloss{SQLite3} 5.1.6 per il salvataggio dei messaggi scambiati tra utente e chatbot. & Obbligatorio & Interna \\
    \hline
    RVO-15 & Dev'essere utilizzato il framework \ccgloss{Tailwind} 3.3.0 per codificare lo stile del front-end. & Obbligatorio & Interna \\
    \hline
    RVO-16 & Dev'essere utilizzato \ccgloss{MinIO} 8.4.3 per l'archiviazione dei documenti. & Obbligatorio & Proponente, Interna \\
    \hline
    RVO-17 & Dev'essere utilizzato \ccgloss{Docker} 24.0.7 per lo sviluppo del prodotto. & Obbligatorio & Interna \\
    \hline
    \rowcolor{white} \caption{Requisiti di vincolo del prodotto}
    \label{tab:reqvin}
\end{xltabular}
\endgroup

\section{Requisiti prestazionali}

\begingroup
\setlength{\tabcolsep}{10pt}
\renewcommand{\arraystretch}{1.5}
\rowcolors{2}{oddrow}{evenrow}
\begin{xltabular}{\textwidth}{| c | X | c | c |}
    \hline
    \rowcolor{headerrow} \textbf{\textcolor{white}{Codice}} & \textbf{\textcolor{white}{Descrizione}} & \textbf{\textcolor{white}{Classificazione}} & \textbf{\textcolor{white}{Fonte}}\\
    \hline
    \endhead
    RPO-1 & Il sistema deve garantire il suo funzionamento con la presenza a sistema di 1000 documenti, tra i formati supportati.  & Obbligatorio & Proponente\\
    \hline
    RPO-2 & Il sistema deve garantire il suo funzionamento con documenti, tra i formati supportati, di dimensione fino a 500KB. & Obbligatorio & Proponente\\
    \hline
    RPZ-3 & Il sistema deve garantire il suo funzionamento con audio, tra i formati supportati, di dimensione fino a 5MB. & Opzionale & Proponente\\
    \hline
    RPO-4 & Il sistema deve essere utilizzabile correttamente nel browser Google Chrome dalla versione 110 e successive. & Obbligatorio & Proponente, Interna\\
    \hline
    RPO-5 & Il sistema deve essere utilizzabile correttamente nel browser Mozilla Firefox dalla versione 116 e successive. & Obbligatorio & Proponente, Interna\\
    \hline
    RPO-6 & Il sistema deve essere utilizzabile correttamente nel browser Opera dalla versione 96 e successive. & Obbligatorio & Proponente, Interna\\
    \hline
    RPO-7 & Il sistema deve essere utilizzabile correttamente nel browser Microsoft Edge dalla versione 110 e successive. & Obbligatorio & Proponente, Interna\\
    \hline
    RPO-8 & Il sistema deve garantire la creazione di almeno una sessione di conversazione. & Obbligatorio & Proponente\\
    \hline
    RPD-9 & Il sistema deve garantire la creazione di almeno due sessioni di conversazione. & Desiderabile & Interna\\
    \hline
    \rowcolor{white} \caption{Requisiti prestazionali del prodotto}
    \label{tab:reqpre}
\end{xltabular}
\endgroup

\section{Requisiti implementativi}
\begingroup
\setlength{\tabcolsep}{10pt}
\renewcommand{\arraystretch}{1.5}
\rowcolors{2}{oddrow}{evenrow}
\begin{xltabular}{\textwidth}{| c | X | c | c |}
    \hline
    \rowcolor{headerrow} \textbf{\textcolor{white}{Codice}} & \textbf{\textcolor{white}{Descrizione}} & \textbf{\textcolor{white}{Classificazione}} & \textbf{\textcolor{white}{Fonte}}\\
    \hline
    \endhead
    RIO-1 & Il sistema deve processare i documenti che vengono caricati, creandone i loro embedding. & Obbligatorio & Interna\\
    \hline
    RIO-2 & Il sistema deve salvare, in modo persistente, i documenti caricati a sistema. & Obbligatorio & Proponente\\
    \hline
    RIO-3 & Il sistema deve salvare, in modo persistente, tutte le informazioni relative ai documenti presenti a sistema. & Obbligatorio & Interna\\
    \hline
    RIO-4 & Il sistema deve salvare, in modo persistente, i vettori dei documenti caricati ed embeddizzati dal sistema. & Obbligatorio & Proponente\\
    \hline
    RIO-5 & Il sistema deve interrompere in modo automatico il processo di generazione della risposta ad una domanda, qualora esso dovesse impiegare un tempo superiore ai 30 secondi. & Obbligatorio & Interna\\
    \hline
    RID-6 & Il sistema deve interrompere in modo automatico la trascrizione della domanda via input vocale, qualora non fosse rilevato alcuna voce per 5 secondi. & Desiderabile & Interno\\
    \hline
    RIO-7 & Il sistema deve salvare, in modo persistente, i messaggi scambiati tra utente e chatbot. & Obbligatorio & Interna\\
    \hline
    RIO-8 & Il sistema deve effettuare una ricerca semantica (semantic search) tra la domanda posta dall'utente e i gli embedding dei documenti caricati restituendo il documento, o pagina di esso, inerente alla domanda. & Obbligatorio & Interna\\
    \hline
    RIO-9 & Il sistema deve interrogare il LLM scelto in base alla domanda posta dall'utente e al relativo documento. & Obbligatorio & Interna\\
    \hline
    \rowcolor{white} \caption{Requisiti prestazionali del prodotto}
    \label{tab:reqimp}
\end{xltabular}
\endgroup

\section{Requisiti di sistema}
Tutti i requisiti riportati in precedenza sono da ritenersi validi solo in osservanza di determinati requisiti di sistema.\\
Per poter utilizzare un Large Language Model fornito da OpenAI è necessario il possesso di una chiave API valida, oltre ad una connessione ad Internet.\\
Per l'utilizzo di Ollama è necessaria la sua installazione ed esecuzione tramite l'immagine Docker ufficiale reperibile a \href{https://ollama.ai/blog/ollama-is-now-available-as-an-official-docker-image}{questo link}.
Nell'utilizzo di un LLM locale, le prestazioni dell'applicazione sono altamente influenzate dal sistema che la esegue. Per poter soddisfare i requisiti individuati, è necessario l'utilizzo di un modello che richiede non meno di 16GB di RAM. Ogni sistema che non soddisfa questo requisito, è quindi da ritenersi improprio all'utilizzo dell'applicazione tramite tecnologie in locale, e non può essere garantito il corretto funzionamento del prodotto. Inoltre, le tecnologie individuate per l'utilizzo di LLM locali richiedono l'adozione di un sistema operativo Linux o MacOS.

\section{Riepilogo}
\begingroup
\setlength{\tabcolsep}{10pt}
\renewcommand{\arraystretch}{1.5}
\rowcolors{2}{oddrow}{evenrow}
\begin{xltabular}{\textwidth}{| X | c | c | c |}
    \hline
    \rowcolor{headerrow} \textbf{\textcolor{white}{Requisito}} & \textbf{\textcolor{white}{Obbligatorio}} & \textbf{\textcolor{white}{Desiderabile}} & \textbf{\textcolor{white}{Opzionale}}\\
    \hline
    \endhead
    Funzionale & 20 & 17 & 13 \\
    \hline
    Di qualità & 4 & 0 & 0 \\
    \hline
    Di vincolo & 12 & 3 & 2 \\
    \hline
    Prestazionali & 7 & 1 & 1 \\
    \hline
    Implementativi & 8 & 1 & 0 \\
    \hline
    \cellcolor{headerrow} \textbf{\textcolor{white}{Totale}} & 51 & 22 & 16 \\
    \hline
    \rowcolor{white} \caption{Riepilogo dei requisiti}
    \label{tab:riepilogo}
\end{xltabular}
\endgroup

