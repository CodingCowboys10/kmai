\chapter{Introduzione} \label{cap:intro}
\section{Scopo del documento}
Il documento si fissa l'obiettivo di descrivere dettagliatamente i casi d’uso e i \ccgloss{requisiti} relativi al \ccgloss{progetto} \ccgloss{Knowledge Management} \ccgloss{AI}, ricavati dall’analisi del \ccgloss{capitolato} C1 del \ccgloss{proponente} \ccgloss{AzzurroDigitale} e dal confronto diretto con l'azienda.
\section{Glossario}
Al fine di prevenire ed evitare possibili ambiguità nei termini e acronimi presenti all’interno della documentazione, è stato realizzato un glossario dove sono riportati i relativi significati (vedasi Glossario\_v1.0). All’interno di ogni documento i termini specifici, che quindi hanno una definizione all’interno del Glossario, saranno contrassegnati con una ‘G’ aggiunta a pedice e trascritti in corsivo. Tale prassi sarà rispettata solamente per la prima occorrenza del termine o acronimo.
\section{Riferimenti}
\subsection{Normativi}
\begin{itemize}
    \item Norme\_di\_progetto\_v1.0
    \item Capitolato d'appalto C1: \\ \url{https://www.math.unipd.it/~tullio/IS-1/2023/Progetto/C1.pdf}
    \item Verbali esterni:
        \begin{itemize}
            \item Verbale\_2023\_11\_09\_v1.0;
            \item Verbale\_2023\_11\_15\_v1.0;
            \item Verbale\_2023\_11\_21\_v1.0;
            \item Verbale\_2023\_11\_29\_v1.0;
            \item Verbale\_2023\_12\_06\_v1.0;
            \item Verbale\_2024\_01\_24\_v1.0.
        \end{itemize}
\end{itemize}

\subsection{Informativi}
\begin{itemize}
    \item Slide dell’insegnamento di Ingegneria del Software, in particolare:
        \begin{itemize}
            \item Analisi dei requisiti: \\ \url{https://www.math.unipd.it/~tullio/IS-1/2023/Dispense/T5.pdf};
            \item Diagrammi dei casi d'uso:\\ \url{https://www.math.unipd.it/~rcardin/swea/2022/Diagrammi%20Use%20Case.pdf};
            \item Approfondimenti:\\ \url{https://\ccgloss{github}.com/rcardin/swe-imdb}
        \end{itemize}
\end{itemize}
