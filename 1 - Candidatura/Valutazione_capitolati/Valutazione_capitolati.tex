\documentclass[12pt]{report}
\usepackage[italian]{babel}
\usepackage[headheight=30pt, textheight=590pt, footskip=50pt, includeheadfoot]{geometry}
\usepackage{graphicx}
\usepackage{tikz}
\usepackage{anyfontsize}
\usepackage{fontspec}
\usepackage{booktabs}
\usepackage{tabularx}
\usepackage{caption}
\usepackage{hyperref}
\usepackage{titling}
\usepackage{fancyhdr}
\usepackage{subfig}
\usepackage{colortbl}
\usepackage{pifont}
\usepackage{titlesec}

\usetikzlibrary{positioning, calc}

\pagestyle{fancy}
\renewcommand{\chaptermark}[1]{%
\markboth{#1}{}}

\definecolor{primarycolor}{HTML}{008ba4}
\definecolor{secondarycolor}{HTML}{b2dce3}
\definecolor{color3}{HTML}{ffffff}
\definecolor{colorline}{HTML}{006172}
\definecolor{oddrow}{HTML}{b2dce3}
\definecolor{evenrow}{HTML}{7fc5d1}
\definecolor{headerrow}{HTML}{006f83}
\definecolor{cmarkcolor}{HTML}{00a46b}
\definecolor{xmarkcolor}{HTML}{a41900}

\titleformat{\chapter}[display]
{\normalfont\huge\bfseries}{}{20pt}{\Huge}

\titlespacing*{\chapter}{0pt}{-60pt}{40pt}

\begin{document}

\fancypagestyle{plain}{%
  \fancyhf{}%
  %\fancyhead[L]{\includegraphics[width=0.1\textwidth]{logo.png}}
  %\fancyhead[C]{}
  %\fancyhead[R]{\leftmark}
  \fancyfoot[L]{}
  \fancyfoot[C]{\thepage}
  \fancyfoot[R]{codingcowboys.swe@gmail.com}
  \renewcommand{\headrulewidth}{0pt}
}

\fancyhf{}
\fancyhead[L]{\includegraphics[width=0.1\textwidth]{logo.png}}
\fancyhead[C]{}
\fancyhead[R]{\leftmark}
\fancyfoot[L]{}
\fancyfoot[C]{\thepage}
\fancyfoot[R]{codingcowboys.swe@gmail.com}


\begin{titlepage}
        \thispagestyle{empty}
        \begin{tikzpicture}[remember picture, overlay]
            \draw[fill=color3] (current page.north west) rectangle (current page.south east);
            \draw[fill=primarycolor, primarycolor] (current page.north west) -- (current page.north) -- (1, -28.7) -- (current page.south west);
            \node [draw,fill=secondarycolor,secondarycolor,text=black,minimum width=\paperwidth,minimum height=5.5cm] (rect) at (current page.center) {};
            \node [circle,xshift=-3cm,draw,colorline,fill=white] (circle) at (current page.center)  {\includegraphics[width=6cm]{logoenome.png}};
            \node [xshift=4.7cm,yshift=24cm,draw,color3,text=black] (uni) at (current page.south){\fontsize{32pt}{36pt}\selectfont Università di Padova};
            \node [draw,color3,text=black, below=of uni] (corso1) {\fontsize{28pt}{32pt}\selectfont Corso di Ingegneria};
            \node [yshift=0.75cm,draw,color3,text=black, below=of corso1] (corso2) {\fontsize{28pt}{32pt}\selectfont del Software};
            \node [xshift=3.5cm,yshift=7cm,draw,color3,text=black] (riunione) at (current page.south){\fontsize{40pt}{48pt}\selectfont Valutazione capitolati};
            \node [draw,color3,text=black, below=of riunione] (date) {\fontsize{28pt}{32pt}\selectfont };
        \end{tikzpicture}
    \end{titlepage}

\chapter*{Storia del documento}
\begingroup
\setlength{\tabcolsep}{10pt}
\renewcommand{\arraystretch}{1.5}
\rowcolors{2}{oddrow}{evenrow}
\begin{tabularx}{\textwidth}{| X | X |}
    \hline
    \rowcolor{headerrow} \textbf{\textcolor{white}{Stato del documento}} & \textbf{\textcolor{white}{Responsabile}} \\
    \hline
    Non approvato & \\
    \hline   
\end{tabularx}
\\ \\ \\
\begin{tabularx}{\textwidth}{| l | l | X | X |}
    \hline
    \rowcolor{headerrow} \textbf{\textcolor{white}{Versione}} & \textbf{\textcolor{white}{Data}} & \textbf{\textcolor{white}{Autori}} & \textbf{\textcolor{white}{Descrizione}} \\
    \hline
     1.0 & 2023/10/30 & Francesco Giacomuzzo, Marco Dolzan  & Revisione documento\\
    \hline 
    0.8 & 2023/10/28 & Marco Dolzan  & Redazione capitolo \ref{chapter:9}\\
    \hline
    0.7 & 2023/10/28 & Leonardo Lago  & Redazione capitolo \ref{chapter:2}\\
    \hline
    0.6 & 2023/10/28 & Giovanni Menon  & Redazione capitolo \ref{chapter:3}\\
    \hline
    0.5 & 2023/10/28 & Francesco Giacomuzzo  & Redazione capitolo \ref{chapter:8}\\
    \hline
    0.4 & 2023/10/27 & Andrea Cecchin  & Redazione capitolo \ref{chapter:7}\\
    \hline
    0.3 & 2023/10/27 & Anna Nordio  & Redazione capitoli \ref{chapter:1} e \ref{chapter:5}\\
    \hline
    0.2 & 2023/10/27 & Francesco Ferraioli & Redazione capitoli \ref{chapter:4} e \ref{chapter:6}\\
    \hline
    0.1 & 2023/10/26 &  Andrea Cecchin, Francesco Ferraioli, Leonardo Lago & Creazione del template del documento\\
    \hline
\end{tabularx}
\endgroup

\tableofcontents

\chapter{Capitolato scelto C1 - Knowledge Management AI} \label{chapter:1}

\section{Informazioni generali}
\begin{itemize}
    \item Proponente: AzzurroDigitale
    \item Committente: Prof. Vardanega Tullio \& Prof. Cardin Riccardo
\end{itemize}

\section{Descrizione}
Il capitolato si basa sulla creazione di una piattaforma web dove è possibile caricare, consultare ed eliminare documenti. Inoltre sarà presente un’interfaccia chat per interagire con un motore di intelligenza artificiale, così da poter accedere alle informazioni presenti nei documenti caricati nella piattaforma web.

\section{Dominio}
\subsection{Dominio applicativo}
Il progetto va a inserirsi nel contesto dell’industria 4.0, usando l’AI per semplificare il reperimento di informazioni da parte degli operatori. Il prodotto finale di questo progetto andrà quindi a rivolgersi potenzialmente a qualsiasi azienda, cercando di rendere immediato l’accesso alla documentazione aziendale per ogni tipologia di dipendente. Un utilizzo più specifico di tale piattaforma sarà riservato alle imprese industriali, andando a implementare un sistema in grado di restituire all’operaio le informazioni utili ad utilizzare la tecnologia aziendale in modo accurato e sicuro.

\subsection{Dominio tecnologico}
Per lo sviluppo del progetto vengono proposte dall’azienda alcune tecnologie in quanto già utilizzate internamente. 
Tra le tecnologie consigliate vengono riportate:

\begin{description}
    \item[Langchain:] framework open source utilizzato per interfacciarsi in maniera black-box con il LLM.
    \item[Angular:] framework per lo sviluppo della parte front-end.
    \item[Node.js:] framework JavaScript per interfacciarsi con le API di OpenAi e di LangChain.
    \item[OpenAi:] LLM provider consigliato.
    \item[Chroma DB:] database vettoriale per contenere gli embeddings creati con Langchain che servono per espandere il LLM a disposizione.
\end{description}

\section{Motivazione scelta}
\subsection{Aspetti positivi}
A seguito di discussioni e votazioni sono risultati i seguenti fattori chiave per la scelta di questo capitolato:
\begin{itemize}
    \item L’utilizzo di tecnologie all’avanguardia che sono attualmente oggetto di ricerca.
    \item Disponibilità e serietà dell’azienda emerse durante il colloquio avuto.
    \item Obiettivi e interessi del proponente chiari.
    \item Conoscenze pregresse nell’ambito dell' apprendimento automatico tramite insegnamenti universitari e conoscenze maturate da alcuni componenti durante attività lavorative pregresse.
    \item Dominio applicativo ritenuto di grande importanza e attualità.
    \item Apertura da parte del proponente a idee implementative proposte dal gruppo.

\end{itemize}

\subsection{Criticità}
A seguito di discussioni e votazioni sono risultate le seguenti criticità:
\begin{itemize}
    \item Non tutti all’interno del team hanno familiarità con alcune tecnologie consigliate.
\end{itemize}
\section{Conclusioni}
Il capitolato ha destato immediato entusiasmo all’interno del team. Inoltre, il colloquio con l’azienda proponente è stato estremamente chiaro ed efficace, permettendo di chiarire i dubbi riguardanti il progetto e di proporre alcune funzionalità aggiuntive pensate dal gruppo, accolte positivamente dai referenti di AzzurroDigitale. 
In conclusione, riteniamo questo capitolato il più stimolante tra quelli proposti, sia per l’ambito applicativo in cui va a inserirsi, sia per le tecnologie utilizzate. Dopo il nostro incontro con l’azienda, lo abbiamo confermato come nostra prima scelta. 


\chapter{Capitolato C2 - Sistemi di raccomandazione}\label{chapter:2}
\section{Informazioni generali}
\begin{itemize}
    \item Proponente: Ergon Informatica Srl
    \item Committente: Prof. Vardanega Tullio \& Prof. Cardin Riccardo.
\end{itemize}

\section{Descrizione}
Realizzare un sistema di raccomandazione che guidi le attività dell’azienda suggerendo a quali clienti rivolgere le attività commerciali e di marketing, cercando i migliori clienti target a cui indirizzarle tramite un approccio multi-strategico da parte del sistema creato, combinando più metodi di misurazione del grado di affinità.

\section{Dominio}
\subsection{Dominio applicativo}
Il prodotto di questo progetto si vuole rivolgere a tutte quelle aziende il cui principale core business è relativo alla vendita di una vasta gamma di prodotti a numerosi clienti. %, dove il cliente ha la possibilità di acquistare un'ampia gamma di prodotti dalla medesima azienda. \\
%Per rendere più efficaci le iniziative di marketing delle singole aziende, il prodotto di tale capitolato si prefigge lo scopo di guidare le suddette attività, suggerendo i migliori clienti target di ogni iniziativa. Il sistema deve essere in grado di associare una tipologia di prodotti al singolo cliente, andando a prevedere in modo corretto il grado di affinità cliente-prodotto.
L'obiettivo di questo capitolato è rendere più efficaci le iniziative di marketing delle singole aziende, suggerendo i migliori clienti target per ciascuna iniziativa.
Il sistema, inoltre, deve essere in grado di associare una tipologia di prodotti al singolo cliente, andando a prevedere in modo corretto il grado di affinità cliente-prodotto.

\subsection{Dominio tecnologico}
Per questo progetto, l'azienda richiede che il sistema possa contare su alcuni componenti, ritenuti necessari:
\begin{itemize}
    \item Database relazionale, per la gestione dei dati relativi al comportamento dei clienti. Vengono proposti per tale scopo SQL Server Express, MariaDB e MySQL.
    \item Sistema di raccomandazione, ottenibile tramite la definizione di un modello multi-strategia e le relative fasi di pre-processing dati. Per tale sistema di raccomandazioni, viene suggerito ML.NET come framework machine learning per il linguaggio C\#, con la comunicazione con il database possibile tramite Entity Framework per ADO.NET. In alternativa può essere utilizzata la libreria Surprise per Python, con l'utilizzo di una fonte dati ODBC per comunicare con il database.
    \item Interfaccia utente, per la consultazione dei risultati prodotti dal sistema e del ritorno dei feedback degli utenti. Essa può essere implementata desktop-based, attraverso WinForms in caso di utilizzo di .NET, o in alternativa web-based.
\end{itemize}

\section{Aspetti positivi e criticità}
\subsection{Aspetti positivi}
A seguito di discussioni e votazioni gli aspetti positivi di questo capitolo sono risultati essere:
\begin{itemize}
    \item Chiarezza dei documenti forniti, che descrivono ed illustrano il problema in modo semplice ed intuitivo.
\end{itemize}

\subsection{Criticità}
A seguito di discussioni e votazioni sono risultate le seguenti criticità:
\begin{itemize}
    \item Ipotizzata complessità di un sistema di raccomandazione multi-strategia basato su differenti modelli di apprendimento automatico.
    \item Il dominio applicativo non è di grande interesse per il gruppo.
\end{itemize}

\section{Conclusioni}
Il progetto, alla luce di riflessioni e valutazioni interne del gruppo, risulta essere uno dei capitolati meno apprezzati tra quelli proposti. Ciò è da associare principalmente ad uno scarso interesse del gruppo rispetto al dominio applicativo del problema presentato.

\chapter{Capitolato C3 - Easy meal}\label{chapter:3}
\section{Informazioni generali}
\begin{itemize}
    \item Proponente: Imola Informatica
    \item Committente: Prof. Vardanega Tullio \& Prof. Cardin Riccardo.
\end{itemize}

\section{Descrizione}
Realizzazione di un'applicazione, indirizzata all'ambito della ristorazione, in grado di raggruppare le funzionalità di consultazione, prenotazione , valutazione, comunicazione, personalizzazione dei piatti e pagamento in un ristorante, tutto in un unico portale.

\section{Dominio}
\subsection{Dominio applicativo}
Il capitolato va a inserirsi nel contesto di un mondo in costante evoluzione verso la digitalizzazione e la personalizzazione delle esperienze. Il prodotto finale di questo progetto è destinato ai diversi attori all'interno del settore della ristorazione e ai consumatori finali, con l'obiettivo di facilitare il processo di prenotazione e di portare una maggiore efficienza nella fase di ordinazione. Il sistema permette all'utente di gestire le proprie prenotazioni in maniera completa e intuitiva, consentendo di interfacciarsi direttamente con i vari ristoranti garantendo così una massima personalizzazione della esperienza finale.

\subsection{Dominio tecnologico}

L'obbiettivo finale del progetto è lo sviluppo di un'applicazione web responsive disponibile sia in ambiente desktop che mobile (Ios, Android). Per lo sviluppo del progetto l'azienda, durante un colloquio, ha consigliato alcune tecnologie come: 
\begin{description}
    \item[React:] Libreria di Javascript per lo sviluppo di interfacce utente.
    \item[Angular:] Framework per lo sviluppo front-end di applicazioni.
\end{description}
L'azienda durante il colloquio ha sottolineato come non ci sono vincoli nelle tecnologie e nei framework da usare. 

\section{Aspetti positivi e criticità}
\subsection{Aspetti positivi}
A seguito di discussioni e valutazioni, i fattori positivi di questo capitolato risultano essere:
\begin{itemize}
  \item Il dominio applicativo è risultato essere interessante e accattivante per il gruppo.
  \item Il dominio tecnico lascia molto spazio a nostre idee e tecnologie.
  \item Supporto e disponibilità da parte dell'azienda.
  \item Obiettivi e interessi spiegati in modo molto dettagliato dal proponente.
  \item Il progetto propone funzionalità non complicate.
\end{itemize}

\subsection{Criticità}
A seguito di discussioni e votazioni, sono risultate le seguenti criticità:
\begin{itemize}
  \item Anche se non complicate, le varie funzionalità sono molto numerose.
  \item Molti casi d'uso e molte funzionalità portano ad una elevata necessità di testing.
\end{itemize}

\section{Conclusioni}
Il progetto, alla luce di riflessioni e votazioni interne del gruppo, risulta essere uno dei tre capitolati ritenuti più interessanti. Questo è dovuto soprattutto alla libertà che l'azienda ha concesso nello scegliere le tecnologie e nel modo di approcciarsi al progetto. Nonostante questo, l'onerosità delle funzionalità ha portato il gruppo a non scegliere questo capitolato.



\chapter{Capitolato C4 - A ChatGPT plugin with Nuvolaris}\label{chapter:4}
\section{Informazioni generali}
\begin{itemize}
    \item Proponente: Nuvolaris
    \item Committente: Prof. Vardanega Tullio \& Prof. Cardin Riccardo.
\end{itemize}

\section{Descrizione}
Il capitolato si basa sulla realizzazione di un plugin ChatGPT, integrante Nuvolaris come piattaforma di appoggio, per la creazione di applicazioni mediante Kubernetes, immediatamente fruibili tramite l'utilizzo di template predefiniti.

\section{Dominio}
\subsection{Dominio applicativo}
Il progetto richiede la realizzazione di un plugin con il quale poter creare in modo facile e veloce un'applicazione. Attraverso una richiesta a ChatGPT, l'utilizzatore finale del prodotto deve poter ottenere un template di codice predefinito, in grado di soddisfare la richiesta iniziale dell'utente. Una volta generato il template dell'applicazione richiesta, deve essere possibile aggiornare/ricostruire tale applicazione grazie a un file di configurazione che si basa sulle richieste a ChatGPT dello user stesso. Il progetto nel suo insieme va ad imitare il sistema di WordPress per la costruzione di pagine Web applicandolo invece alla costruzione di applicazioni.

\subsection{Dominio tecnologico}
L'intero progetto ha come centro del proprio dominio tecnologico l'utilizzo della piattaforma Nuvolaris, sistema incloud nel quale saranno generate le applicazioni richieste dall'utente attraverso un plugin in ChatGPT.\\
In particolare, le risposte ad ogni domanda dell'utente dovranno essere degli url dove poter ritrovare l'app appena costruita. Per fare ciò, la prima richiesta sarà la creazione di una app appartenente ad una categoria ben precisa. Le successive richieste porteranno invece all'evoluzione continua del prodotto reperibile su Nuvolaris.

\section{Aspetti positivi e criticità}
\subsection{Aspetti positivi}
I fattori positivi di questo capitolato risultano essere:
\begin{itemize}
  \item L’utilizzo di tecnologie all’avanguardia che sono attualmente oggetto di ricerca.
  \item Il proponente fornisce abbonamenti e servizi necessari allo svolgimento del capitolato.
\end{itemize}

\subsection{Criticità}
I fattori critici di questo capitolato risultano essere:
\begin{itemize}
    \item Scarsa chiarezza espositiva del capitolato.
    \item Scarsa chiarezza sulle tecnologie necessarie alla realizzazione del suddetto progetto.
    \item Complessità apparente del progetto decisamente elevata.
    \item L'elevata aleatorietà di ChatGPT rapportata ad un sistema richiedente l'unione di template che devono coesistere sembra essere di difficile implementazione.
\end{itemize}

\section{Conclusioni}
Il progetto, alla luce di riflessioni e votazioni interne del gruppo, risulta essere il capitolato meno apprezzato tra quelli proposti, causa da imputare principalmente all'esposizione poco chiara del capitolato.

\chapter{Capitolato C5 - WMS3: warehouse management 3D}\label{chapter:5}
\section{Informazioni generali}
\begin{itemize}
    \item Proponente: Sanmarco Informatica SPA
    \item Committente: Prof. Vardanega Tullio \& Prof. Cardin Riccardo.
\end{itemize}

\section{Descrizione}
Il capitolato consiste nella realizzazione di un sistema di gestione tridimensionale di magazzini, con il quale ottimizzare la disposizione dei prodotti di un magazzino. In questo sistema dovrà essere permessa la realizzazione della struttura dell'ambiente e lo spostamento del materiale, potendo identificare la dislocazione dei materiali nei vari magazzini, controllare la loro movimentazione e gestire processi come ricevimento, spedizione e utilizzo nei reparti produttivi.

\section{Dominio}
\subsection{Dominio applicativo}
Il sistema deve poter permettere la creazione dell'ambiente del magazzino in tre dimensioni, così come tutti gli oggetti presenti in esso. Dev'essere possibile spostare in modo facile e veloce ogni singolo elemento, ottenendo così una analisi del flusso dei materiali attraverso delle simulazioni. Il sistema si rivolge a tutte le aziende interessate a rendere efficiente la gestione dei propri magazzini, attraverso il monitoraggio dello stato del magazzino e l'ottimizzazione della fase di comprensione, progettazione, simulazione e utilizzo dell'ambiente. Un tipico caso d'uso del prodotto è presentato dai magazzini di IKEA, dove attraverso delle postazioni posizionate nel perimetro dell'ambiente è possibile identificare la locazione esatta di un prodotto.\\
Il prodotto finale richiesto prevede sessioni volatili, dunque non è prevista alcuna persistenza dati, oltre alla sola gestione lato "amministratore", senza alcuna funzione di login.

\subsection{Dominio tecnologico}
Per la realizzazione del progetto viene caldamente consigliato l'utilizzo di una specifica piattaforma:
\begin{description}
    \item[Three.js:] piattaforma cross-browser per la creazione e visualizzazione di contenuti grafici 3D in un web browser.
\end{description} 
Come alternativa, vengono consigliati altri motori grafici per lo sviluppo di contenuti interattivi e animazioni 3D in tempo reale quali Unity e Unreal Engine.
\section{Aspetti positivi e criticità}
\subsection{Aspetti positivi}
I fattori positivi di questo capitolato risultano essere:
\begin{itemize}
  \item Il proponente raccomanda contatti frequenti tra azienda e gruppo per confronti.
  \item Il dominio applicativo è risultato essere molto interessate per il gruppo.
\end{itemize}

\subsection{Criticità}
I fattori critici di questo capitolato risultano essere:
\begin{itemize}
  \item Le tecnologie consigliate sono sconosciute a tutti i membri del gruppo e non sembrano di facile apprendimento.
\end{itemize}

\section{Conclusioni}
Il progetto alla luce di riflessioni e votazioni interne del gruppo si posiziona a metà nella classifica di gradimento del gruppo. Questo è dovuto principalmente alle tecnologie consigliate completamente sconosciute al gruppo e ritenute costose in termini di tempo impiegato per assumere dimestichezza con esse.

\chapter{Capitolato C6 - SyncCity: Smart city monitoring platform}\label{chapter:6}
\section{Informazioni generali}
\begin{itemize}
    \item Proponente: Sync Lab
    \item Committente: Prof. Vardanega Tullio \& Prof. Cardin Riccardo.
\end{itemize}

\section{Descrizione}
Il capitolato consiste nello sviluppare una piattaforma che riesca a simulare o raccogliere grandi quantitativi di dati riguardanti lo stato di salute di una città, rappresentandoli in una serie di dashboard. Analizzando tali dati sarà possibile prendere decisioni veloci ed efficaci.


\section{Dominio}
\subsection{Dominio applicativo}
Il prodotto finale consiste in una piattaforma contenente una serie di dashboard, nelle quali poter visualizzare le principali informazioni che caratterizzano lo stato di salute di una città: temperatura, umidità, qualità dell'aria, livello del traffico, locazione di parcheggi liberi, locazione di lavori di manutenzione della rete stradale ecc...\\
Tale piattaforma si vuole quindi rivolge potenzialmente ad ogni città dotata di sensori in grado di fornire le informazioni sul proprio stato di salute.

\subsection{Dominio tecnologico}
Per il corretto svolgimento del capitolato, sono fortemente consigliate le seguenti tecnologie:
\begin{itemize}
  \item Utilizzo di framework per la simulazione dei dati, preferendo l'utilizzo di script e librerie di generazione casuale di dati in Python.
  \item Utilizzo di Apache Kafka come gestore per il gathering di dati.
  \item Utilizzo di database colonnari OLAP, come ClickHouse, in grado di mantenere la persistenza dati con elevata numerosità.
  \item Utilizzo di Grafana come piattaforma di data visualization.
\end{itemize}

\section{Aspetti positivi e criticità}
\subsection{Aspetti positivi}
I fattori positivi di questo capitolato risultano essere:
\begin{itemize}
    \item Chiarezza nella descrizione del progetto e delle caratteristiche.
    \item Risulta essere un progetto al passo con i tempi e di implementazione reale.
    \item Tecnologie necessarie specificate dal proponente, il quale si propone anche di effettuare formazione su di esse.
\end{itemize}

\subsection{Criticità}
I fattori critici di questo capitolato risultano essere:
\begin{itemize}
    \item La gestione, immagazzinamento e uso di grosse moli di dati risulta essere un task complesso, il che ha sollevato qualche perplessità tra i membri del gruppo.
    \item Nessun componente ha mai utilizzato le tecnologie necessarie per la realizzazione di questo capitolato.
\end{itemize}

\section{Conclusioni}
Il progetto, alla luce di riflessioni e votazioni interne del gruppo, risulta essere il quarto progetto più apprezzato: nel complesso risulta essere un progetto ritenuto fattibile e positivo e l'azienda si è dimostrata essere disponibile.


\chapter{Capitolato C7 - ChatGPT vs BedRock developer analysis}\label{chapter:7}
\section{Informazioni generali}
\begin{itemize}
    \item Proponente: Zero12
    \item Committente: Prof. Vardanega Tullio \& Prof. Cardin Riccardo.
\end{itemize}

\section{Descrizione}
Creazione di un middleware che riceva in input dei requisiti di business e produca epic e user stories associate tramite ChatGPT e AWS BedRock, comparando la capacità delle due IA nell’interpretare il codice sorgente ed associare le user stories generate.

\section{Dominio}
\subsection{Dominio applicativo}
Tramite il prodotto di questo progetto, l'utente finale deve poter inserire in input, tramite un'interfaccia web, dei requisiti di business. Successivamente, attraverso sia ChatGPT che BedRock, saranno generate user stories ed epic stories a seguito dell'interpretazione del codice sorgente. Dopo aver salvato i dati su un database, sarà poi possibile visualizzare ed analizzare tali risultati nella stessa interfaccia web. Interesse primario è comparare e analizzare le diverse interpretazioni del codice sorgente, svolte da ChatGPT e BedRock.

\subsection{Dominio tecnologico}
Oltre all'uso necessario di ChatGPT e BedRock, per la realizzazione del progetto sono fortemente raccomandate le tecnologie di Amazon Web Service come:
\begin{description}
    \item[AWS Fargate:] per la gestione a container.
    \item[MongoDB:] tecnologia per database non relazionali.
\end{description}
Come linguaggi di programmazione si identifica l'utilizzo di:
\begin{description}
    \item[Node.Js:] per lo sviluppo di API.
    \item[Python:] per lo sviluppo del plugin per Xcode richiesto.
    \item[Typescript:] per lo sviluppo del plugin per Visual Studio Code richiesto.
\end{description}
L'intera architettura dovrà inoltre essere basata a micro-servizi.

\section{Aspetti positivi e criticità}
\subsection{Aspetti positivi}
I fattori positivi di questo capitolato risultano essere:
\begin{itemize}
    \item Il proponente si è dichiarato disponibile nell'erogare piccoli corsi di formazione sull'uso delle tecnologie richieste.
    \item La fase di analisi dei requisiti è sostenuta dalla volontà dell'azienda proponente di svolgere sedute di design thinking e brainstorming per aiutarci.
    \item Il tema di questo capitolato risulta essere attuale per il tipo di tecnologie coinvolte.
\end{itemize}

\subsection{Criticità}
I fattori critici di questo capitolato risultano essere:
\begin{itemize}
    \item Ampia complessità del progetto, frutto della necessità di sviluppare più plugin e funzionalità, tutte con tecnologie specifiche differenti.
    \item Alcune delle tecnologie necessarie sono sconosciute ai più, o non raggiungibili da alcun membro del gruppo, come Apple Xcode.
    \item Elevato numero di requisiti e funzionalità richieste al prodotto finale.
\end{itemize}

\section{Conclusioni}
Il progetto, a seguite di riflessioni, non risulta aver riscosso l'attenzione del gruppo, preoccupato dall'ampio numero di tecnologie richieste, molte di queste poco o per nulla familiari ai componenti del gruppo. La complessità del prodotto finale di tale capitolato, e il tempo necessario alla realizzazione nella sua totalità, hanno portato il gruppo a scartare questa proposta.

\chapter{Capitolato C8 - JMAP: il nuovo protocollo per la posta elettronica}\label{chapter:8}
\section{Informazioni generali}
\begin{itemize}
    \item Proponente: Zextras
    \item Committente: Prof. Vardanega Tullio \& Prof. Cardin Riccardo.
\end{itemize}

\section{Descrizione}
In questo capitolato è richiesto lo sviluppo di una dimostrazione eseguibile di un servizio di e-mail applicando il protocollo JMAP, in modo tale da valutarne le prestazioni, la manutenibilità e la completezza. Lo scopo di questa dimostrazione è fornire all'azienda delle motivazioni e degli strumenti di paragone per valutare l'integrazione del protocollo JMAP a Carbonio, software sviluppato da Zextras, attualmente basato su IMAP.

\section{Dominio}
\subsection{Dominio applicativo}
Il prodotto finale consiste in un client e-mail, la cui architettura è estendibile al di fuori del suo scopo principale. È richiesto infatti lo sviluppo una serie di funzionalità riguardanti non solo lo scambio di e-mail, ma anche la gestione di cartelle condivise; il che comprende creazione, eliminazione e condivisione di queste cartelle, e inserimento e rimozione di oggetti al loro interno. 

\subsection{Dominio tecnologico}
Al fine di realizzare questo progetto, sono stati indicati i seguenti domini tecnologici
\begin{description}
    \item[Java:] Questo linguaggio costituisce il nucleo principale di Carbonio e verrà utilizzato per lo sviluppo del progetto;
    \item[iNPUTmice/jmap:] sarà la libreria impiegata per l’implementazione del protocollo JMAP.
    \item[Docker:] in quanto il progetto deve essere realizzato in un container. Questa scelta è consigliata anche per effettuare test in maniera più efficiente.
    \item[Nodi stateless:] per garantire una buona scalabilità del servizio proposto.
\end{description}

\section{Aspetti positivi e criticità}

\subsection{Aspetti positivi}
Gli aspetti positivi relativi a questo capitolato sono:
\begin{itemize}
    \item Sono stati forniti link a documentazioni delle principali tecnologie da utilizzare.
\end{itemize}
\subsection{Criticità}
\begin{itemize}
    \item Tecnologie consigliate non familiari ai membri del gruppo.
    \item Alcune documentazioni non sono ancora complete.
\end{itemize}
\section{Conclusioni}
Dopo un'attenta considerazione e discussione all'interno del gruppo, è emerso che questo progetto non ha suscitato un interesse significativo. Le nostre preoccupazioni riguardano principalmente le tecnologie da usare, le quali sono sconosciute ai membri del gruppo e non suscitano una notevole curiosità.

\chapter{Capitolato C9 - ChatSQL: creare frasi SQL da linguaggio naturale}\label{chapter:9}
\section{Informazioni generali}
\begin{itemize}
    \item Proponente: Zucchetti
    \item Committente: Prof. Vardanega Tullio \& Prof. Cardin Riccardo.
\end{itemize}

\section{Descrizione}
Sviluppare un'applicazione in grado di generare un prompt, da fornire a ChatGPT, combinando la descrizione della struttura del database con una richiesta in linguaggio naturale. La risposta di ChatGPT al prompt finale deve essere la query risolutrice della domanda posta in linguaggio naturale.

\section{Dominio}
\subsection{Dominio applicativo}
Il prodotto finale consiste in una applicazione che, data una interrogazione in linguaggio naturale fatta da un utente a un database, deve poter restituire un prompt che sia combinazione della struttura del database con la richiesta fatta dall’utente, così che sia possa fornire il prompt a ChatGPT e avere come risposta una query nel linguaggio SQL.

\subsection{Dominio tecnologico}
Per il corretto svolgimento del capitolato, non sono consigliate tecnologie specifiche per la realizzazione e gestione del database contenente le informazioni a cui l'utente dovrà accedere.
Sono invece stati consigliati diversi tipi di LLM da poter utilizzare per la verifica del prompt generato dall’applicazione, come ad esempio chatGPT.


\section{Aspetti positivi e criticità}
\subsection{Aspetti positivi}
Gli aspetti positivi che ci hanno fatto includere questo capitolato nelle nostre tre preferenze sono:
\begin{itemize}
    \item Possibilità di utilizzare tecnologie innovative e in continua evoluzione.
    \item Conoscenze pregresse da parte di alcuni membri del gruppo riguardanti le tecnologie impiegate in questo progetto.
    \item Il proponente si è dichiarato disponibile a interagire col gruppo e aiutarlo per tutta la durata del progetto.
\end{itemize}

\subsection{Criticità}
A seguito di discussioni e votazioni sono risultate le seguenti criticità:
\begin{itemize}
    \item Non tutti all’interno del team hanno familiarità con alcune tecnologie consigliate.
\end{itemize}

\section{Conclusioni}
Il capitolato è rientrato fin da subito tra i tre preferiti dal team in quanto molto stimolante e interessante per l’ambito applicativo e tecnologico in cui va a inserirsi. Inoltre, il colloquio con l’azienda proponente è stato chiaro ed efficace, permettendo di chiarire i dubbi riguardanti il progetto. Nonostante ciò, si è preferito scegliere un altro capitolato perché ritenuto più coinvolgente per il nostro gruppo.


\end{document}
