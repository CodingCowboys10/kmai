\documentclass[12pt]{article}
\usepackage[italian]{babel}
\usepackage[margin=80pt,headheight=50pt,includeheadfoot]{geometry}
\usepackage{graphicx} % Required for inserting images
\usepackage{tikz}
\usepackage{anyfontsize}
\usepackage{fontspec}
\usepackage{booktabs}
\usepackage{tabularx}
\usepackage{caption}
\usepackage{hyperref}
\usepackage{titling}
\usepackage{fancyhdr}
\usepackage{subfig}
\usepackage{colortbl}
\usepackage{pifont}

\usetikzlibrary{positioning, calc}

\pagestyle{fancy}

\definecolor{primarycolor}{HTML}{008ba4}
\definecolor{secondarycolor}{HTML}{b2dce3}
\definecolor{color3}{HTML}{ffffff}
\definecolor{colorline}{HTML}{006172}
\definecolor{oddrow}{HTML}{b2dce3}
\definecolor{evenrow}{HTML}{7fc5d1}
\definecolor{headerrow}{HTML}{006f83}
\definecolor{cmarkcolor}{HTML}{00a46b}
\definecolor{xmarkcolor}{HTML}{a41900}

\begin{document}
\fancyhf{}
\fancyhead[L]{\includegraphics[width=0.1\textwidth]{logo.png}}
\fancyhead[C]{Verbale riunione}
\fancyhead[R]{Coding Cowboys}
\fancyfoot[L]{19 ottobre 2023}
\fancyfoot[C]{\thepage}
\fancyfoot[R]{codingcowboys.swe@gmail.com}

\begin{titlepage}
        \thispagestyle{empty}
        \begin{tikzpicture}[remember picture, overlay]
            \draw[fill=color3] (current page.north west) rectangle (current page.south east);
            \draw[fill=primarycolor, primarycolor] (current page.north west) -- (current page.north) -- (1, -28.7) -- (current page.south west);
            \node [draw,fill=secondarycolor,secondarycolor,text=black,minimum width=\paperwidth,minimum height=5.5cm] (rect) at (current page.center) {};
            \node [circle,xshift=-3cm,draw,colorline,fill=white] (circle) at (current page.center)  {\includegraphics[width=6cm]{logoenome2.png}};
            \node [xshift=4.7cm,yshift=24cm,draw,color3,text=black] (uni) at (current page.south){\fontsize{32pt}{36pt}\selectfont Università di Padova};
            \node [draw,color3,text=black, below=of uni] (corso1) {\fontsize{28pt}{32pt}\selectfont Corso di Ingegneria};
            \node [yshift=0.75cm,draw,color3,text=black, below=of corso1] (corso2) {\fontsize{28pt}{32pt}\selectfont del Sofware};
            \node [xshift=3.5cm,yshift=7cm,draw,color3,text=black] (riunione) at (current page.south){\fontsize{40pt}{48pt}\selectfont Verbale Riunione};
            \node [draw,color3,text=black, below=of riunione] (date) {\fontsize{28pt}{32pt}\selectfont 19 ottobre 2023};
        \end{tikzpicture}
    \end{titlepage}

\section{Informazioni della riunione} \label{sec:info}
La riunione è stata svolta in presenza dalle ore 14:30 alle ore 17:00

\subsection{Lista partecipanti} \label{subsec:partecipanti}
\begingroup
    \setlength{\tabcolsep}{10pt}
    \renewcommand{\arraystretch}{1.5}
    \rowcolors{2}{oddrow}{evenrow}
    \begin{tabular}{| l | l | c |}
        \hline
        \rowcolor{headerrow}\textbf{\textcolor{white}{Partecipante}} & \textbf{\textcolor{white}{Ruolo}} & \textbf{\textcolor{white}{Presenza}} \\
        \hline
        Andrea Cecchin & Redattore & \textcolor{cmarkcolor}{\ding{51}}\\
        \hline
        Marco Dolzan & Verificatore & \textcolor{cmarkcolor}{\ding{51}}\\
        \hline
        Francesco Ferraioli & Verificatore & \textcolor{cmarkcolor}{\ding{51}}\\
        \hline
        Francesco Giacomuzzo & Verificatore & \textcolor{cmarkcolor}{\ding{51}}\\
        \hline
        Leonardo Lago & Redattore & \textcolor{cmarkcolor}{\ding{51}}\\
        \hline
        Giovanni Menon & Amministratore & \textcolor{cmarkcolor}{\ding{51}}\\
        \hline
        Anna Nordio & Responsabile & \textcolor{cmarkcolor}{\ding{51}}\\
        \hline
    \end{tabular}
\endgroup
\label{tab:partecipanti}
%\newpage

\section{Ordine del giorno} \label{sec:agenda}
\subsection{Discussioni} \label{subsec:discussione}
\begin{enumerate}
    \item Concordare i canali di comunicazione informali
    \item Scelta del nome e logo del gruppo
    \item Creazione dell'indirizzo di posta elettronica del gruppo
    \item Divisione dei ruoli per le seguenti 2 settimane
    \item Discussione sulle preferenze dei capitolati proposti 
\end{enumerate}

\subsection{Votazioni} \label{subsec:votazione}
\begin{enumerate}
    \item Finalizzare la scelta del nome del gruppo
    \item Finalizzare la scelta del logo del gruppo
\end{enumerate}
%\newpage
\section{Resoconto} \label{sec:resoconto}
\subsection{Discussioni} \label{subsec:resdiscussione}
\begin{enumerate}
    \item Si sono stabili come canali di comunicazione informali, chat di gruppo su Telegram, sever Discord e Notion workspace
    \item Si sono raccolte le varie proposte scaturite nei canali di comunicazione informali sovracitati sia per il nome che per il logo
    \item Si è creato l'indirizzo di posta elettronica codingcowboys.swe@gmail.com. 
    \item Si sono stabiliti i ruoli ricoperti da ciascun componente del gruppo che hanno validità immediata fino all'aggiudicazione di un capitolato. Si veda la tabella nella sottosezione \ref{tab:partecipanti}
    \item Si sono raccolte le opinioni di ciascun membro del gruppo tenendo in considerazione le diverse discussioni avvenute nei canali di comunicazione informali sovracitati. Si è quindi creato un foglio elettronico in cui ognuno ha espresso una valutazione da 1 a 9 mutualmente esclusiva (link al foglio elettronico: \url{https://docs.google.com/spreadsheets/d/1s50pfQHOXVb5-U64ijQavQbHquG-dzEn2njDACqJriE/edit?usp=sharing}). La media dei voti per ciascun capitolato ha così generato una classifica delle preferenza che vede come prime 3 scelte rispettivamente i capitolati numero 1, 9 e 3. Per ognuno di essi si è quindi stilata una lista di domande da porre al proponente riguardo argomenti poco chiari presenti nei capitolati (link alla lista delle domande raccolte: \url{https://docs.google.com/document/d/1b3XOG-xtldQIatobF35Gn50k5Un1x1iHAcsYn0fE4HY/edit?usp=sharing}). Si è quindi incaricato Giovanni Menon di programmare l'invio di una e-mail ai proponenti dei capitoli ai quali il gruppo è interessato, con la richiesta di un colloquio per il chiarimento dei sopracitati dubbi sorti.
    
\end{enumerate}

\subsection{Votazioni} \label{subsec:resvotazione}
\begin{enumerate}
    \item Esito votazione del nome del gruppo: \\\\
        \begingroup
            \setlength{\tabcolsep}{10pt}
            \renewcommand{\arraystretch}{1.5}
            \rowcolors{2}{oddrow}{evenrow}
            \begin{tabularx}{0.93\textwidth}{| l | l | X |}
                 \hline
                 \rowcolor{headerrow}\textbf{\textcolor{white}{Proposta}} & \textbf{\textcolor{white}{Sommario}} & \textbf{\textcolor{white}{Mittente}} \\
                 \hline
                 Coding Cowboys & 100\%  & Andrea Cecchin, Marco Dolzan, Francesco Ferraioli, Francesco Giacomuzzo, Leonardo Lago, Giovanni Menon, Anna Nordio \\
                 \hline
                 HeSWEnberg & 0\% & \\
                 \hline
                 Astenuti & 0\% & \\
                 \hline
            \end{tabularx}
        \endgroup
    \item Esito votazione logo del gruppo: \\\\
        \begingroup
            \setlength{\tabcolsep}{10pt}
            \renewcommand{\arraystretch}{1.5}
            \rowcolors{2}{oddrow}{evenrow}
            \begin{tabularx}{0.93\textwidth}{| l | l | X |}
                 \hline
                 \rowcolor{headerrow}\textbf{\textcolor{white}{Proposta}} & \textbf{\textcolor{white}{Sommario}} & \textbf{\textcolor{white}{Mittente}} \\
                 \hline
                 Figura \ref{fig:logo1} & 100\%  & Andrea Cecchin, Marco Dolzan, Francesco Ferraioli, Francesco Giacomuzzo, Leonardo Lago, Giovanni Menon, Anna Nordio \\
                 \hline
                 Figura \ref{fig:logo2} & 0\% & \\
                 \hline
                 Astenuti & 0\% & \\
                 \hline
            \end{tabularx}
        \endgroup
        \begin{figure}
            \centering
            \subfloat[][\label{fig:logo1}]{
            \includegraphics[width=0.4\textwidth]{logoenome.png}} \quad
            \subfloat[][\label{fig:logo2}]{
            \includegraphics[width=0.3\textwidth]{logobrutto.png}} \quad
            \caption{Loghi proposti}
            \label{fig:loghi}
        \end{figure}
\end{enumerate}

\subsection{Prossima riunione} \label{subsec:riunione}
Ogni partecipante ha dato la propria disponibilità a partecipare ad almeno una riunione settimanale che si terrà alle ore 14:30 e avrà durata variabile.  La prossima riunione è quindi fissata per il giorno 26 ottobre 2023 alle ore 14:30. 

%\newpage
\section{Storia del documento} \label{sec:storia}
\begingroup
\setlength{\tabcolsep}{10pt}
\renewcommand{\arraystretch}{1.5}
\rowcolors{2}{oddrow}{evenrow}
\begin{tabularx}{\textwidth}{| l | l | X | X |}
    \hline
    \rowcolor{headerrow} \textbf{\textcolor{white}{Versione}} & \textbf{\textcolor{white}{Data}} & \textbf{\textcolor{white}{Autori}} & \textbf{\textcolor{white}{Descrizione}} \\
    \hline
    0.1 & 2023/10/18 & Andrea Cecchin, Leonardo Lago  & Creazione del template documento. Compilate sezioni \ref{sec:info} e \ref{sec:agenda}\\
    \hline
    1.0 & 2023/10/19 & Andrea Cecchin, Leonardo Lago &  Aggiornata tabella nella sottosezione \ref{subsec:partecipanti}. Compilate la sezioni \ref{sec:resoconto}\\
    \hline
    1.1 & 2023/10/21 & Andrea Cecchin, Leonardo Lago & Modificato lo stile delle tabelle e della prima pagina\\
    \hline
    1.2 & 2023/10/21 & Francesco Ferraioli & Correzione errori ortografici\\
    \hline
    1.3 & 2023/10/21 & Marco Dolzan, Francesco Ferraioli, Francesco Giacomuzzo & Revisione documento\\
    \hline
\end{tabularx}
\endgroup
\end{document}
