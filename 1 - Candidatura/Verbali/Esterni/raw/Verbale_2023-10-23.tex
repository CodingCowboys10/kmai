\documentclass[12pt]{article}
\usepackage[italian]{babel}
\usepackage[headheight=33pt, footskip=50pt]{geometry}
\usepackage{graphicx}
\usepackage{tikz}
\usepackage{anyfontsize}
\usepackage{fontspec}
\usepackage{booktabs}
\usepackage{tabularx}
\usepackage{caption}
\usepackage{hyperref}
\usepackage{titling}
\usepackage{fancyhdr}
\usepackage{subfig}
\usepackage{colortbl}
\usepackage{pifont}

\usetikzlibrary{positioning, calc}

\pagestyle{fancy}

\definecolor{primarycolor}{HTML}{008ba4}
\definecolor{secondarycolor}{HTML}{b2dce3}
\definecolor{color3}{HTML}{ffffff}
\definecolor{colorline}{HTML}{006172}
\definecolor{oddrow}{HTML}{b2dce3}
\definecolor{evenrow}{HTML}{7fc5d1}
\definecolor{headerrow}{HTML}{006f83}
\definecolor{cmarkcolor}{HTML}{00a46b}
\definecolor{xmarkcolor}{HTML}{a41900}

\begin{document}
\fancyhf{}
\fancyhead[L]{\includegraphics[width=0.1\textwidth]{logo.png}}
\fancyhead[C]{Verbale esterno}
\fancyhead[R]{Coding Cowboys}
\fancyfoot[L]{23 ottobre 2023}
\fancyfoot[C]{\thepage}
\fancyfoot[R]{codingcowboys.swe@gmail.com}

\begin{titlepage}
        \thispagestyle{empty}
        \begin{tikzpicture}[remember picture, overlay]
            \draw[fill=color3] (current page.north west) rectangle (current page.south east);
            \draw[fill=primarycolor, primarycolor] (current page.north west) -- (current page.north) -- (1, -28.7) -- (current page.south west);
            \node [draw,fill=secondarycolor,secondarycolor,text=black,minimum width=\paperwidth,minimum height=5.5cm] (rect) at (current page.center) {};
            \node [circle,xshift=-3cm,draw,colorline,fill=white] (circle) at (current page.center)  {\includegraphics[width=6cm]{logoenome.png}};
            \node [xshift=4.7cm,yshift=24cm,draw,color3,text=black] (uni) at (current page.south){\fontsize{32pt}{36pt}\selectfont Università di Padova};
            \node [draw,color3,text=black, below=of uni] (corso1) {\fontsize{28pt}{32pt}\selectfont Corso di Ingegneria};
            \node [yshift=0.75cm,draw,color3,text=black, below=of corso1] (corso2) {\fontsize{28pt}{32pt}\selectfont del Sofware};
            \node [xshift=3.5cm,yshift=7cm,draw,color3,text=black] (riunione) at (current page.south){\fontsize{40pt}{48pt}\selectfont Verbale esterno};
            \node [draw,color3,text=black, below=of riunione] (date) {\fontsize{28pt}{32pt}\selectfont 23 ottobre 2023};
            \node [draw,color3,text=black, below=of date, xshift=-5cm] (resp) {\fontsize{20pt}{24pt}\selectfont Responsabile:};
            \node [right=of resp] (firma) {\includegraphics[width=5cm]{Firma_Anna.png}};
        \end{tikzpicture}
    \end{titlepage}

\section{Informazioni della riunione} \label{sec:info}
La riunione è stata svolta da remoto, tramite la piattaforma Zoom, dalle ore 15:15 alle ore 15:55, in presenza di Gregorio Piccoli, un rappresentante dell'azienda Zucchetti.

\subsection{Lista partecipanti} \label{subsec:partecipanti}
\begingroup
    \setlength{\tabcolsep}{10pt}
    \renewcommand{\arraystretch}{1.5}
    \rowcolors{2}{oddrow}{evenrow}
    \begin{tabular}{| l | l | c |}
        \hline
        \rowcolor{headerrow}\textbf{\textcolor{white}{Partecipante}} & \textbf{\textcolor{white}{Ruolo}} & \textbf{\textcolor{white}{Presenza}} \\
        \hline
        Andrea Cecchin & Redattore & \textcolor{cmarkcolor}{\ding{51}}\\
        \hline
        Marco Dolzan & Verificatore & \textcolor{cmarkcolor}{\ding{51}}\\
        \hline
        Francesco Ferraioli & Verificatore & \textcolor{cmarkcolor}{\ding{51}}\\
        \hline
        Francesco Giacomuzzo & Verificatore & \textcolor{cmarkcolor}{\ding{51}}\\
        \hline
        Leonardo Lago & Redattore & \textcolor{cmarkcolor}{\ding{51}}\\
        \hline
        Giovanni Menon & Amministratore & \textcolor{cmarkcolor}{\ding{51}}\\
        \hline
        Anna Nordio & Responsabile & \textcolor{cmarkcolor}{\ding{51}}\\
        \hline
    \end{tabular}
    \endgroup
\label{tab:partecipanti}
\\ \\In rappresentanza di Zucchetti:\\ \\
\begingroup
    \setlength{\tabcolsep}{10pt}
    \renewcommand{\arraystretch}{1.5}
    \rowcolors{2}{oddrow}{evenrow}
    \begin{tabular}{|l|}
        \hline
        \rowcolor{headerrow}\textbf{\textcolor{white}{Partecipante}} \\
        \hline
        Gregorio Piccoli\\
        \hline
    \end{tabular}
\endgroup

\section{Ordine del giorno} \label{sec:agenda}
\subsection{Discussioni} \label{subsec:discussione}
Discussione di dubbi riguardanti il capitolato proposto dall'azienda, "ChatSQL", tramite il format domanda-risposta.


\section{Resoconto} \label{sec:resoconto}
\subsection{Discussioni} \label{subsec:resdiscussione}
Nel corso dell'incontro, sono state poste le seguenti domande al signor Piccoli:
\begin{enumerate}
    \item Relativamente alla pagina 3 di presentazione del capitolato, cosa si intende con Calibration Bias Prompting, e più in generale quale prompt ci si aspetta dal processo di generazione?\\ \\
Il signor Piccoli ha spiegato che con il termine prompt ci si riferisce banalmente a come viene scritta la domanda rivolta a ChatGPT. Nel tentativo di fornire un esempio chiarificatore, viene fornita una dimostrazione di come ChatGPT risponde a un prompt contenente delle informazioni, una struttura minimale di un database e una richiesta in linguaggio naturale. L'output della dimostrazione è la riorganizzazione delle informazioni inserite basandosi sulla struttura del database fornita, riportando in linguaggio SQL la query richiesta in linguaggio naturale. In particolare, attraverso un processo di Prompt Engineering, si vuole arrivare all'ottimizzazione del prompt finale da eseguire su ChatGPT, assicurandosi che quest'ultimo contenga solo le informazioni relative al dizionario dati strettamente necessarie per permettere la generazione della query richiesta.
    \item Relativamente a quanto riportato nei requisiti del capitolato, cosa si intende che le funzionalità del generatore di prompt devono essere integrate in un sistema unico? \\ \\
Viene spiegato che questo vincolo indica la necessità che le funzionalità desiderate, ovvero la generazione di un prompt in risposta all'inserimento del dizionario dati e della richiesta in linguaggio naturale, siano "confezionate" e facilmente testabili all'interno di una applicazione.
    \item Quali tecnologie e librerie consigliate per la realizzazione di questo progetto? \\ \\
Vengono segnalati i siti \url{https://chat.lmsys.org/} e \url{https://huggingface.co/} come fonte open-source utile di modelli di machine learning. Viene inoltre segnalata la libreria di ricerca semantica txtai \url{https://github.com/neuml/txtai}. Il resto delle tecnologie e librerie da utilizzare sono lasciate alla libera ricerca del gruppo.
    \item Cosa è pronta a fornire l'azienda in nostro supporto nel corso del progetto?\\ \\
L'azienda, in particolar modo con la figura di Piccoli Gregorio quale riferimento, è pienamente disponibile a realizzare videoconferenze con il gruppo previa richiesta via e-mail. In alternativa, sarà facilmente realizzabile un incontro in presenza presso la sede Zucchetti a Padova.
    
\end{enumerate}

\section{Storia del documento} \label{sec:storia}
\begingroup
\setlength{\tabcolsep}{10pt}
\renewcommand{\arraystretch}{1.5}
\rowcolors{2}{oddrow}{evenrow}
\begin{tabularx}{\textwidth}{| l | l | X | X |}
    \hline
    \rowcolor{headerrow} \textbf{\textcolor{white}{Versione}} & \textbf{\textcolor{white}{Data}} & \textbf{\textcolor{white}{Autori}} & \textbf{\textcolor{white}{Descrizione}} \\
    \hline
    1.0 & 2023/10/26 & Marco Dolzan , Francesco Ferraioli , Francesco Giacomuzzo & Verifica verbale \\
    \hline
    0.3 & 2023/10/24 & Francesco Giacomuzzo & Aggiornata sezione \ref{sec:info}\\
    \hline
    0.2 & 2023/10/23 & Andrea Cecchin, Leonardo Lago & Redazione sezione \ref{sec:resoconto}. Aggiornata sezione \ref{sec:info}\\
    \hline
    0.1 & 2023/10/22 & Andrea Cecchin, Leonardo Lago  & Redazione sezioni \ref{sec:info} e \ref{sec:agenda}\\
    \hline
\end{tabularx}  
\endgroup
\end{document}
