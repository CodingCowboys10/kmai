\documentclass[12pt]{article}
\usepackage[italian]{babel}
\usepackage[headheight=30pt, textheight=590pt, footskip=50pt, includeheadfoot]{geometry}
\usepackage{graphicx}
\usepackage{tikz}
\usepackage{anyfontsize}
\usepackage{fontspec}
\usepackage{booktabs}
\usepackage{tabularx}
\usepackage{caption}
\usepackage{hyperref}
\usepackage{titling}
\usepackage{fancyhdr}
\usepackage{subfig}
\usepackage{colortbl}
\usepackage{pifont}

\usetikzlibrary{positioning, calc}

\pagestyle{fancy}

\definecolor{primarycolor}{HTML}{008ba4}
\definecolor{secondarycolor}{HTML}{b2dce3}
\definecolor{color3}{HTML}{ffffff}
\definecolor{colorline}{HTML}{006172}
\definecolor{oddrow}{HTML}{b2dce3}
\definecolor{evenrow}{HTML}{7fc5d1}
\definecolor{headerrow}{HTML}{006f83}
\definecolor{cmarkcolor}{HTML}{00a46b}
\definecolor{xmarkcolor}{HTML}{a41900}


\begin{document}
\fancyhf{}
\fancyhead[L]{\includegraphics[width=0.1\textwidth]{logo.png}}
\fancyhead[C]{}
\fancyhead[R]{Preventivo costi e assunzione impegni}
\fancyfoot[L]{}
\fancyfoot[C]{\thepage}
\fancyfoot[R]{codingcowboys.swe@gmail.com}

\begin{titlepage}
        \thispagestyle{empty}
        \begin{tikzpicture}[remember picture, overlay]
            \draw[fill=color3] (current page.north west) rectangle (current page.south east);
            \draw[fill=primarycolor, primarycolor] (current page.north west) -- (current page.north) -- (1, -28.7) -- (current page.south west);
            \node [draw,fill=secondarycolor,secondarycolor,text=black,minimum width=\paperwidth,minimum height=5.5cm] (rect) at (current page.center) {};
            \node [circle,xshift=-3cm,draw,colorline,fill=white] (circle) at (current page.center)  {\includegraphics[width=6cm]{logoenome.png}};
            \node [xshift=4.7cm,yshift=24cm,draw,color3,text=black] (uni) at (current page.south){\fontsize{32pt}{36pt}\selectfont Università di Padova};
            \node [draw,color3,text=black, below=of uni] (corso1) {\fontsize{28pt}{32pt}\selectfont Corso di Ingegneria};
            \node [yshift=0.75cm,draw,color3,text=black, below=of corso1] (corso2) {\fontsize{28pt}{32pt}\selectfont del Software};
            \node [xshift=3.5cm,yshift=7cm,draw,color3,text=black] (riunione) at (current page.south){\fontsize{40pt}{48pt}\selectfont Preventivo costi e };
            \node [xshift=3.5cm,yshift=5.5cm,draw,color3,text=black] (riunione) at (current page.south){\fontsize{40pt}{48pt}\selectfont assunzione impegni};
            \node [draw,color3,text=black, below=of riunione] (date) {\fontsize{28pt}{32pt}\selectfont };
        \end{tikzpicture}
    \end{titlepage}

%%%changelog
\begingroup
\pagestyle{plain}

\begin{huge}
    \textbf{Storia del documento} \\ \\ \\ \\
\end{huge}
\begingroup
\setlength{\tabcolsep}{10pt}
\renewcommand{\arraystretch}{1.5}
\rowcolors{2}{oddrow}{evenrow}
\begin{tabularx}{\textwidth}{| X | X |}
    \hline
    \rowcolor{headerrow} \textbf{\textcolor{white}{Stato del documento}} & \textbf{\textcolor{white}{Responsabile}} \\
    \hline
    Non approvato & \\
    \hline
\end{tabularx}
\\ \\ \\
\begin{tabularx}{\textwidth}{| l | l | X | X |}
    \hline
    \rowcolor{headerrow} \textbf{\textcolor{white}{Versione}} & \textbf{\textcolor{white}{Data}} & \textbf{\textcolor{white}{Autori}} & \textbf{\textcolor{white}{Descrizione}} \\
    \hline
    1.0 & 2023/10/31 & Marco Dolzan, Francesco Giacomuzzo & Verifica documento \\
    \hline
    0.4 & 2023/10/31 & Francesco Giacomuzzo & Aggiornamento sezioni \ref{sec:impegnoOra} e \ref{sec:considerazioniSR} \\
    \hline
    0.3 & 2023/10/28 & Andrea Cecchin, Francesco Ferraioli, Francesco Giacomuzzo, Leonardo Lago, Giovanni Menon, Anna Nordio & Aggiornamento sezione \ref{sec:considerazioniSR}, redazione sezioni \ref{sec:Preventivo} e \ref{sec:Scadenza} \\
    \hline
    0.2 & 2023/10/26 & Andrea Cecchin, Marco Dolzan, Francesco Ferraioli, Francesco Giacomuzzo, Leonardo Lago, Giovanni Menon, Anna Nordio & Redazione sezioni \ref{sec:impegnoOra} e \ref{sec:considerazioniSR} \\
    \hline
    0.1 & 2023/10/26 &  Andrea Cecchin, Francesco Ferraioli, Leonardo Lago & Creazione del template del documento\\
    \hline
\end{tabularx}
\endgroup

\cleardoublepage
\endgroup
%%%

\section{Impegno orario} \label{sec:impegnoOra}
La seguente tabella intende esporre il quantitativo di ore che ogni membro dovrà, in media, spendere nel corso del progetto in un determinato ruolo. Combinando questi dati con il costo orario di ogni diversa figura, sarà possibile determinare, quanto più accuratamente possibile, un preventivo dei costi.
\begingroup
\setlength{\tabcolsep}{10pt}
\renewcommand{\arraystretch}{1.5}
\rowcolors{2}{oddrow}{evenrow}
\begin{table}[h]
    \centering
    \begin{tabularx}{\textwidth}{| l | l | X | l |}
        \hline
        \rowcolor{headerrow} \textbf{\textcolor{white}{Ruolo}} & \textbf{\textcolor{white}{Costo a ora}} & \textbf{\textcolor{white}{Ore ruolo}} & \textbf{\textcolor{white}{Ore membro}} \\
        \hline
        Responsabile \textbf{(Res.)}& 30 & 56  & 8\\
        \hline
        Amministratore \textbf{(Amm.)}& 20 & 56  & 8\\
        \hline
        Progettista \textbf{(Prj.)}& 25 & 126  & 18\\
        \hline
        Analista \textbf{(Ana.)}& 25 & 91  & 13\\
        \hline
        Programmatore \textbf{(Prg.)}& 15 & 154  & 22\\
        \hline
        Verificatore \textbf{(Ver.)}& 15 & 168  & 24\\
        \hline
        & Totale: 13055€ & Totale: 651 & Totale: 93\\
        \hline
    \end{tabularx}
    \caption{Costo orario per ruolo}
    \label{tab:costi}
\end{table}
\\Viene inoltre prevista la seguente distribuzione oraria dei ruoli per ogni membro del gruppo:
\begin{table}[h!]
    \centering
    \begin{tabularx}{\textwidth}{| l | X | X | X | X | X | X |}
        \hline
        \rowcolor{headerrow} \textbf{\textcolor{white}{Membro}} & \textbf{\textcolor{white}{Res.}} & \textbf{\textcolor{white}{Amm.}} & \textbf{\textcolor{white}{Prj.}} & \textbf{\textcolor{white}{Ana.}} & \textbf{\textcolor{white}{Prg.}} & \textbf{\textcolor{white}{Ver.}} \\
        \hline
        Anna Nordio & 9 & 8  & 19 & 13 & 21 & 23 \\
        \hline
        Giovanni Menon & 7 & 7  & 20 & 12 & 24 &  23\\
        \hline
        Leonardo Lago & 8 & 8  & 18 & 14 & 22 &  23\\
        \hline
        Marco Dolzan & 7 & 8  & 16 & 13 & 24 & 25 \\
        \hline
        Francesco Ferraioli & 9 & 10  & 17 & 11 & 22 & 24\\
        \hline
        Francesco Giacomuzzo & 8 & 8  & 18 & 14 & 22 & 23\\
        \hline
        Andrea Cecchin & 8 & 7  & 18 & 14 & 20 & 26\\
        \hline
    \end{tabularx}
    \caption{Suddivisione ore per membro}
    \label{tab:ore}
\end{table}
\endgroup

\section{Considerazioni sui ruoli} \label{sec:considerazioniSR}
I diversi quantitativi orari ritenuti necessari per ciascun ruolo sono frutto delle seguenti osservazioni e considerazioni:
    \subsection{Responsabile ed Amministratore} \label{subsec:Resp&Amm}
Il responsabile, o project manager, è la figura preposta a controllare e coordinare l'operato del gruppo, curando le relazioni verso l'esterno del progetto: è il ruolo su cui si condensa la responsabilità dell'andamento del progetto. L'amministratore invece è colui che definisce e supervisiona il dominio informatico e tecnologico necessario alle normali mansioni di ogni membro del gruppo: è lui a decidere e a mettere a disposizione le risorse utili al way of working collettivo.\\
Considerando che tali ruoli, secondo la nostra analisi del progetto, necessitano di meno ore produttive di lavoro per portare a termine i propri compiti, riteniamo che tali figure richiedano un quantitativo orario molto inferiore ad ogni altro.

    \subsection{Analista} \label{subsec:Ana}
Chi ricopre il ruolo di analista è chiamato a ricoprire un ruolo fondamentale per l'analisi e la corretta comprensione dei problemi relativi al progetto.\\
Nonostante tale figura non sarà necessaria nell'intero arco temporale del progetto, riteniamo sia necessario un ampio quantitativo orario nella prima parte del progetto, per assicurare una corretta analisi dei requisiti del capitolato.

    \subsection{Progettista} \label{subsec:proj}
Il progettista è colui che decreta le scelte realizzative per rendere possibile quanto emerso dal lavoro dagli analisti. Nello specifico, si occupa di progettare l'architettura del software identificando i componenti principali e specificando standard tecnici, strumenti e piattaforme da utilizzare, documentando in modo chiaro la progettazione per permettere al team di sviluppo di agire coerentemente con le specifiche adottate. Deve, inoltre, comunicare al meglio tali scelte in modo da permettere all'intero gruppo di agire conformemente ad esse.\\
Alla luce di una analisi del capitolato, riteniamo che la fase di progettazione richieda un dispendio orario consistente, in quanto fondamentale tanto quanto una corretta analisi dei requisiti per lo sviluppo dell'applicativo.
\newpage
\subsection{Verificatore} \label{subsec:Ver}
Il verificatore è il membro preposto alla verifica della qualità di quanto prodotto dagli altri membri del gruppo. Esso svolge un ruolo fondamentale in quanto garantisce la conformità e la correttezza del prodotto e della documentazione ad esso associata.\\
Tenendo in considerazione la presenza permanente di tale figura per tutto l'arco temporale necessario alla conclusione del progetto, abbiamo ritenuto opportuno assegnare a tale ruolo il maggior numero di ore.
    \subsection{Programmatore} \label{subsec:progrm}
Il programmatore è colui che realizza le scelte implementative dei progettisti e si occupa della manutenzione del prodotto lungo il suo ciclo di vita.\\
Ritenendo la fase di programmazione, relativa al capitolato scelto, più dispendiosa rispetto al lavoro di progettisti e analisti, abbiamo pensato di assegnare al ruolo del programmatore un quantitativo orario superiore alle suddette mansioni. Questa scelta è motivata dalla presenza costante di tale figura, il cui lavoro è necessario per raggiungere i requisiti individuati in fase di analisi nelle modalità descritte in fase di progettazione.

\section{Preventivo dei Costi} \label{sec:Preventivo}
Il costo finale del progetto, in accordo a quanto già visibile nella Tabella \ref{tab:costi}, ammonta a 13055€. Tale cifra è da intendersi come costo preventivo per portare a termine il capitolato, basandosi sulle considerazioni della sezione \ref{sec:considerazioniSR}.
\section{Scadenza di Consegna} \label{sec:Scadenza}
Viene previsto, come data ultima di consegna del prodotto relativo al capitolato proposto da AzzurroDigitale, il giorno 5 Aprile 2024. Tale data è stata decisa tenendo in considerazione il modello di 20 settimane predisposto da AzzurroDigitale e comunicatoci durante il colloquio avuto.
\begingroup
\endgroup

\end{document}
